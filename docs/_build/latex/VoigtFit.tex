%% Generated by Sphinx.
\def\sphinxdocclass{report}
\documentclass[letterpaper,10pt,english]{sphinxmanual}
\ifdefined\pdfpxdimen
   \let\sphinxpxdimen\pdfpxdimen\else\newdimen\sphinxpxdimen
\fi \sphinxpxdimen=.75bp\relax

\usepackage[utf8]{inputenc}
\ifdefined\DeclareUnicodeCharacter
 \ifdefined\DeclareUnicodeCharacterAsOptional
  \DeclareUnicodeCharacter{"00A0}{\nobreakspace}
  \DeclareUnicodeCharacter{"2500}{\sphinxunichar{2500}}
  \DeclareUnicodeCharacter{"2502}{\sphinxunichar{2502}}
  \DeclareUnicodeCharacter{"2514}{\sphinxunichar{2514}}
  \DeclareUnicodeCharacter{"251C}{\sphinxunichar{251C}}
  \DeclareUnicodeCharacter{"2572}{\textbackslash}
 \else
  \DeclareUnicodeCharacter{00A0}{\nobreakspace}
  \DeclareUnicodeCharacter{2500}{\sphinxunichar{2500}}
  \DeclareUnicodeCharacter{2502}{\sphinxunichar{2502}}
  \DeclareUnicodeCharacter{2514}{\sphinxunichar{2514}}
  \DeclareUnicodeCharacter{251C}{\sphinxunichar{251C}}
  \DeclareUnicodeCharacter{2572}{\textbackslash}
 \fi
\fi
\usepackage{cmap}
\usepackage[T1]{fontenc}
\usepackage{amsmath,amssymb,amstext}
\usepackage{babel}
\usepackage{times}
\usepackage[Bjarne]{fncychap}
\usepackage{sphinx}

\usepackage{geometry}

% Include hyperref last.
\usepackage{hyperref}
% Fix anchor placement for figures with captions.
\usepackage{hypcap}% it must be loaded after hyperref.
% Set up styles of URL: it should be placed after hyperref.
\urlstyle{same}

\addto\captionsenglish{\renewcommand{\figurename}{Fig.}}
\addto\captionsenglish{\renewcommand{\tablename}{Table}}
\addto\captionsenglish{\renewcommand{\literalblockname}{Listing}}

\addto\captionsenglish{\renewcommand{\literalblockcontinuedname}{continued from previous page}}
\addto\captionsenglish{\renewcommand{\literalblockcontinuesname}{continues on next page}}

\addto\extrasenglish{\def\pageautorefname{page}}

\setcounter{tocdepth}{1}



\title{VoigtFit Documentation}
\date{Feb 26, 2018}
\release{0.9.1}
\author{Jens-Kristian Krogager}
\newcommand{\sphinxlogo}{\vbox{}}
\renewcommand{\releasename}{Release}
\makeindex

\begin{document}

\maketitle
\sphinxtableofcontents
\phantomsection\label{\detokenize{index::doc}}


VoigtFit is written to make absorption line fitting a \sphinxstyleemphasis{breeze}.
Dig in, and you will be constraining the contents of metals in remote galaxies before you know it!

Development of \sphinxstyleemphasis{VoigtFit} happens \sphinxhref{https://github.com/jkrogager/VoigtFit}{on GitHub} so if you encounter any issues or you have
ideas for great new features, you can \sphinxhref{https://github.com/jkrogager/VoigtFit/issues}{raise an issue on the GitHub page}.


\chapter{Installation instructions}
\label{\detokenize{index:voigtfit}}\label{\detokenize{index:installation-instructions}}

\section{Installation}
\label{\detokenize{install:installation}}\label{\detokenize{install::doc}}\label{\detokenize{install:install}}
VoigtFit is currently only written and tested for Python 2.7;
However, the code is currently being ported to Python 3.6 \textendash{} Stay tuned!


\subsection{Dependencies}
\label{\detokenize{install:dependencies}}
VoigtFit depends on \sphinxhref{https://matplotlib.org/}{matplotlib}, \sphinxhref{http://www.numpy.org/}{numpy}, \sphinxhref{https://scipy.org/}{scipy}, \sphinxhref{http://www.h5py.org/}{h5py}, \sphinxhref{http://www.astropy.org/}{astropy}, and \sphinxhref{https://lmfit.github.io/lmfit-py/}{lmfit}. You
can install these using your favorite Python package manager such as
\sphinxhref{https://pip.pypa.io/en/stable/installing/}{pip} or
\sphinxhref{http://conda.pydata.org/docs/}{conda}.


\subsection{Using pip}
\label{\detokenize{install:using-pip}}
The easiest way to install the most recent stable version of \sphinxcode{\sphinxupquote{VoigtFit}} is
using \sphinxhref{http://www.pip-installer.org/}{pip}:

\fvset{hllines={, ,}}%
\begin{sphinxVerbatim}[commandchars=\\\{\}]
\PYGZpc{}\PYG{o}{]} pip install VoigtFit
\end{sphinxVerbatim}


\subsection{From source}
\label{\detokenize{install:from-source}}
Alternatively, you can get the source by downloading a \sphinxhref{https://github.com/jkrogager/VoigtFit/tarball/master}{tarball} or cloning \sphinxhref{https://github.com/jkrogager/VoigtFit}{the git
repository}:

\fvset{hllines={, ,}}%
\begin{sphinxVerbatim}[commandchars=\\\{\}]
\PYGZpc{}\PYG{o}{]} git clone https://github.com/jkrogager/VoigtFit.git
\end{sphinxVerbatim}

Once you’ve downloaded the source, you can navigate to the root
directory and run:

\fvset{hllines={, ,}}%
\begin{sphinxVerbatim}[commandchars=\\\{\}]
\PYGZpc{}\PYG{o}{]} python setup.py install
\end{sphinxVerbatim}


\subsection{Test the installation}
\label{\detokenize{install:h5py}}\label{\detokenize{install:test-the-installation}}
If the installation went smoothly, you should be able to run VoigtFit from the terminal
by excecuting the following command:

\fvset{hllines={, ,}}%
\begin{sphinxVerbatim}[commandchars=\\\{\}]
\PYGZpc{}\PYG{o}{]} VoigtFit
\end{sphinxVerbatim}

Running the program without any input will create an empty parameter file template.
This way you can always set up a fresh parameter file when starting a new project.
Moreover, as the program grows and more features are implemented, a comment about
such new features will appear automatically in the parameter file template.
This way you can stay updated on what is possible within the parameter file language.

To run the program with a given input file, simply excecute the command:

\fvset{hllines={, ,}}%
\begin{sphinxVerbatim}[commandchars=\\\{\}]
\PYGZpc{}\PYG{o}{]} VoigtFit  input.pars
\end{sphinxVerbatim}

Where input.pars is the name of your parameter file.


\chapter{Documentation}
\label{\detokenize{index:documentation}}

\section{VoigtFit Parameter Language}
\label{\detokenize{documentation::doc}}\label{\detokenize{documentation:voigtfit-parameter-language}}
\begin{sphinxadmonition}{note}{Note:}
The telluric template was obtained from ESOs \sphinxhref{http://www.eso.org/observing/etc/skycalc}{skycalc}.

The dataset is first initiated with data defined through a “data” statement. This sets the spectral resolution. If the spectral resolution subsequently needs to be changed, this should be done using a “resolution” statement.

The transitions that should be fitted are then defined through “lines” statements and the component structure is defined through “component” statements (see also “copy component” and “delete component”). Note that components must be defined for all the ions that are defined in the “lines” statements.

If the data are not already normalized, and the chebyshev polynomial fitting has been turned off (see ‘C\_order’ below), a window for each line fitting region will pop up allowing the user to normalize the data by selecting a left and right continuum region. This will fit a straight line to the continuum and normalize the region. Alternatively, the user can specify a ‘norm\_method’ = spline to select a set of spline points used for the continuum estimation.

After the continuum normalization is done, the user can define spectral masks for each line fitting region. For each region, the user must select left and right boundaries by clicking with the mouse in the plotting window. This will mask out the region in between the two boundaries, and the data in the masked region will therefore not be used in the fit. The user can define as many masked ranges as desired. When the masking for a given line is done, the user must confirm the mask by clicking enter in the terminal. Note that if an uneven number of boundaries are given, the ranges are invalid and the user must select the ranges again.
The masking step can be skipped by including the ‘nomask’ statement in the parameter file. Or can be run for individual lines by using the ‘mask’ statement.

Lastly, the fit will be performed and the resulting best-fit parameters will be printed to the terminal. If the “save” statement is given in the parameter file, the parameters will also be saved to a file and the resulting best-fit profiles will be saved to a pdf file. Otherwise, the best-fit solution is plotted in an interactive window.

If the “abundance“ or “metallicity” statements are given, the total abundance for each element or its metallicity relative to Solar is printed to the terminal.
\end{sphinxadmonition}

The parameter language allows the user to define parameters without using python scripting
which has a slightly more complex syntax. There are a few general rules for the parameter language:
\begin{quote}

Everything that comes after a \sphinxtitleref{\#} sign is regarded as a comment and is not parsed.

The number of spaces in a given line between parameters and values is not important.

The order of the statements is not important either. This will be handled by the program
at runtime, so you can freely arrange the parameter file as you see fit.

All optional arguments below are stated in square brackets.

Double underscores below indicate an optional keyword argument to be given by the user.
\end{quote}


\subsection{The Basics}
\label{\detokenize{documentation:the-basics}}
In the following the basic and mandatory statements are presented.
If these are not present in the parameter file, or if their values are
not understood, the program will cause an error.

\begin{sphinxadmonition}{note}{Note:}
An parameter file template can be created in your working directory by running \sphinxcode{\sphinxupquote{VoigtFit}} with no arguments.
\end{sphinxadmonition}

The first part of the parameter file usually defines a bit of metadata for the dataset.
These variables are as follows:


\subsubsection{name}
\label{\detokenize{documentation:name}}
\sphinxstylestrong{name :  dataset\_name}
\begin{quote}

\sphinxstyleemphasis{dataset\_name} gives the name of the dataset.
The dataset is automatically saved (as \sphinxstyleemphasis{‘dataset\_name.hdf5’}),
and if a dataset of the given name is present, it will be loaded automatically.
\end{quote}


\subsubsection{save}
\label{\detokenize{documentation:save}}
\sphinxstylestrong{save : {[} filename {]}}

\sphinxtitleref{filename} is the filename used for the graphic output and for parameter output.
If no filename is given, the dataset \sphinxtitleref{name} attribute will be used.
The graphic output will be saved in pdf format and the parameter files will be saved as ascii files:
The best-fit parameters will be saved to \sphinxtitleref{filename.fit} and the best-fit continuum parameters will be saved to \sphinxtitleref{filename}.cont.


\subsubsection{z\_sys}
\label{\detokenize{documentation:z-sys}}
\sphinxstylestrong{z\_sys :  z\_sys}
\begin{quote}

\sphinxstyleemphasis{z\_sys} gives the systemic redshift of the absorption system.
Relative velocities are calculated with respect to this redshift.
\end{quote}


\subsubsection{data}
\label{\detokenize{documentation:data}}
\sphinxstylestrong{data  filename  resolution  {[} norm   air {]}}
\begin{quote}

\sphinxstyleemphasis{filename} specifies the path to the spectrum
(should be an ASCII table with up to four columns: wavelength, flux, error, mask).

\sphinxstyleemphasis{resolution}  is the spectral resolution of the given spectrum in units of km/s.
\end{quote}

Optional arguments:
\begin{quote}

\sphinxstyleemphasis{norm} : if present in the line, this indicates that the spectrum in filename are normalized.

\sphinxstyleemphasis{air} : if present, the wavelengths in the spectrum will be converted from air to vacuum.
\end{quote}

\begin{sphinxShadowBox}
\sphinxstyletopictitle{Example}

\sphinxcode{\sphinxupquote{data  'J2350-0052\_uvb.tab'  40.  air}}
\begin{quote}

This will load the data from the file named ‘J2350-0052\_uvb.tab’,
convert the wavelength column from air to vacuum, and assign
a spectral resolution of 40 km/s.
\end{quote}

\sphinxcode{\sphinxupquote{data  "norm\_data/norm\_2350-0052\_vis.tab"  32.7  norm}}
\begin{quote}

This will load the data from the file named ‘norm\_2350-0052\_vis.tab’
in the directory ‘norm\_data’ and assign a spectral resolution of 32.7 km/s.
The keyword ‘norm’ is present, so the data will be marked as normalized,
and no interactive normalization will therefore pop up during data preparation.
\end{quote}
\end{sphinxShadowBox}


\subsubsection{lines}
\label{\detokenize{documentation:lines}}
\sphinxstylestrong{lines  line\_tags  {[} velspan=\_\_ {]}}
\begin{quote}

\sphinxstyleemphasis{line\_tags} can be a single line or multiple lines separated by blank spaces.
The line tag should match a line in the line-list, e.g., FeII\_2374, SiII\_1526,
or HI\_1215. For the Lyman series of hydrogen and deuterium, the following
notation is also accepted: HI\_1 for the Ly-alpha, HI\_3 for Ly-gamma, and so on.
\end{quote}

Optinal arguments:
\begin{quote}

\sphinxstyleemphasis{velspan} : the value after the equal-sign is taken as the velocity
span in km/s around each line to be defined as a fitting region.
The default span is 500 km/s.
\end{quote}

\begin{sphinxShadowBox}
\sphinxstyletopictitle{Example}

\sphinxcode{\sphinxupquote{lines  FeII\_2260  FeII\_2374  SiII\_1808  HI\_1215}}
\begin{quote}

This will define the two singly ionized iron transitions at 2260 and 2374Å
together with the singly ionized silicon transition at 1808Å and the Ly-alpha line.
\end{quote}

\sphinxcode{\sphinxupquote{lines FeII\_2374  SiII\_1808}}
\begin{quote}

This will define the iron and silicon lines with default velocity spans.
\end{quote}

\sphinxcode{\sphinxupquote{lines HI\_1  HI\_2  velspan=5000}}
\begin{quote}

This will define the Ly-\(\alpha\) and Ly-\(\beta\) lines with a larger 5000 km/s velocity span.
\end{quote}
\end{sphinxShadowBox}


\subsubsection{molecule}
\label{\detokenize{documentation:molecule}}
\sphinxstylestrong{molecule  element  bands  {[} J=\_\_  velspan=\_\_ {]}}
\begin{quote}

\sphinxstyleemphasis{element} refers to the molecule to be fitted, for now only CO and H2 are defined
in the database.
\begin{description}
\item[{\sphinxstyleemphasis{bands} designates a list of vibrational bands for the given molecule.}] \leavevmode
For CO: the A(\(\nu\)) -\textgreater{} A(0) bands for \(\nu\) up to \(\nu\)=11, the C(0) -\textgreater{} X(0) band, the d(5) -\textgreater{} X(0)
and e(1) -\textgreater{} X(0). The bands are referred to as AX(\(\nu\)-0), CX(0-0), dX(5-0), and eX(1-0).

For H$_{\text{2}}$: the Lyman bands B(\(\nu\)) -\textgreater{} X(0) for \(\nu\) up to \(\nu\)=19 (BX(\(\nu\)-0)) and Werner bands
C(\(\nu\)) -\textgreater{} X(0) for \(\nu\) up to \(\nu\)=5 (CX(\(\nu\)-0)).

\end{description}
\end{quote}

Optional arguments:
\begin{quote}

\sphinxstyleemphasis{J} : the upper rotational level to include for the given bands.
All rotational levels from \sphinxstyleemphasis{J=0} up to (and including) \sphinxstyleemphasis{J} will be included.
For CO the maximum \sphinxstyleemphasis{J} level included in the database is \sphinxstyleemphasis{J=4}, for H$_{\text{2}}$ this is \sphinxstyleemphasis{J=7}.
The default value is \sphinxstyleemphasis{J=1}.

\sphinxstyleemphasis{velspan} : the value after the equal-sign is taken as the velocity
span in km/s around each line to be defined as a fitting region.
The default span is 500 km/s.
\end{quote}

\begin{sphinxShadowBox}
\sphinxstyletopictitle{Example}

\sphinxcode{\sphinxupquote{molecule H2  BX(0-0)  BX(1-0)  BX(2-0)  J=5}}
\begin{quote}

This will define the rotational levels up to \sphinxstyleemphasis{J=5} for the three lowest vibrational
Lyman bands of H$_{\text{2}}$.
\end{quote}

\sphinxcode{\sphinxupquote{molecule CO  AX(0-0)  AX(1-0)  J=4  velspan=120}}
\begin{quote}

This will define the rotational levels up to \sphinxstyleemphasis{J=4} for the two lowest vibrational bands of CO.
\end{quote}
\end{sphinxShadowBox}


\subsubsection{component}
\label{\detokenize{documentation:component}}
\sphinxstylestrong{component  ion  z  b  logN  {[} var\_z=True/False  var\_b=True/False  var\_N=True/False
tie\_z=\_\_  tie\_b=\_\_  tie\_N=\_\_  velocity{]}}

alt.: component  ion  z=\_\_  b=\_\_  logN=\_\_  {[} var\_z=True/False  var\_b=True/False  var\_N=True/False
tie\_z=\_\_  tie\_b=\_\_  tie\_N=\_\_  velocity{]}
\begin{quote}

\sphinxstyleemphasis{ion} specifies for which ion the component should be defined, e.g., FeII, SiII.

\sphinxstyleemphasis{z} gives the redshift of the component.

\sphinxstyleemphasis{b} gives the broadening parameter of the given component.

\sphinxstyleemphasis{logN} gives the 10-base logarithm of the column density for the given component in cm$^{\text{-2}}$.

Note: The order of the values of z, b, logN must be followed, unless the are given
as keyword arguments, i.e., logN=\_\_  z=\_\_  b=\_\_
\end{quote}

Optional arguments:
\begin{quote}

Parameters which should be kept fixed can be set by the optional arguments \sphinxstyleemphasis{fix\_z} for redshift,
\sphinxstyleemphasis{fix\_b} for broadening parameter, and \sphinxstyleemphasis{fix\_N} for column density.
These are passed as keyword values which are either \sphinxstyleemphasis{True} or \sphinxstyleemphasis{False}, the default is \sphinxstyleemphasis{False}.

Parameters for different components and ions can be tied to each other using the
\sphinxstyleemphasis{tie\_z}, \sphinxstyleemphasis{tie\_b}, \sphinxstyleemphasis{tie\_N} options. This is mostly used to tie redshifts or broadening parameters
for different species.
The parameters are tied using the following naming convention:
the name of a given parameter is made up by the \sphinxstyleemphasis{base} (which is either ‘z’, ‘b’, or ‘logN’),
the component \sphinxstyleemphasis{number} (starting from 0), and the \sphinxstyleemphasis{ion} (e.g., FeII).
The \sphinxstyleemphasis{base} and \sphinxstyleemphasis{number} are joined together with no spaces in between,
and the \sphinxstyleemphasis{ion} is appended with an underscore (‘\_’) as in between; e.g., ‘z0\_FeII’ for the first
component of FeII.

\sphinxstyleemphasis{velocity} : if this keyword is included, the first argument (or z=) will be interpreted
as a velocity offset relative to \sphinxstyleemphasis{z\_sys}.
\end{quote}

\begin{sphinxShadowBox}
\sphinxstyletopictitle{Example}

\sphinxcode{\sphinxupquote{component  FeII  1.957643  7.0  14.5  var\_z=False}}
\begin{quote}

This will define a component for FeII at z=1.957643 with \sphinxstyleemphasis{b} = 7.0 km/s and a column density
of 10$^{\text{14.5}}$ cm$^{\text{-2}}$. The redshift will no be varied during the fit.
\end{quote}

\sphinxcode{\sphinxupquote{component  SiII  -109.5  7.0  16.0  tie\_b='b0\_FeII' velocity}}
\begin{quote}

This will define a component for SiII at a relative velocity of -109.5 km/s
with \sphinxstyleemphasis{b} = 7.0 km/s and a column density 10$^{\text{16.0}}$.
The \sphinxstyleemphasis{b}-parameter will be tied to the first component defined for \sphinxstyleemphasis{FeII}.
\end{quote}
\end{sphinxShadowBox}


\subsubsection{interactive components}
\label{\detokenize{documentation:interactive-components}}
\sphinxstylestrong{interactive  line\_tags}
\begin{quote}

\sphinxstyleemphasis{line\_tags} can be a single line or multiple lines separated by blank spaces or commas.
The line tag should match a line in the line-list, e.g., FeII\_2374, SiII\_1526.
The line tag must be defined in the dataset (using the \sphinxcode{\sphinxupquote{lines}} statement).
\end{quote}

This command will activate the interactive window for defining components for the given lines.
Notice that this will overwrite any other components defined previously for this element.

\begin{sphinxShadowBox}
\sphinxstyletopictitle{Example}

Give example screenshot here!
\end{sphinxShadowBox}


\subsubsection{copy components}
\label{\detokenize{documentation:copy-components}}
\sphinxstylestrong{copy components from ion1 to ion2  {[} (scale  logN  ref\_comp)  tie\_z=True/False  tie\_b=True/False {]}}
\begin{quote}

\sphinxstyleemphasis{ion1} denotes the \sphinxstyleemphasis{ion} from which to copy components (FeII, CI, etc.).
Components must manually be defined for this ion.

\sphinxstyleemphasis{ion2} denotes the \sphinxstyleemphasis{ion} to which the components will be copied.
Lines must be defined for this \sphinxstyleemphasis{ion} using a \sphinxcode{\sphinxupquote{lines}} statement.

Note: The order is not important. This is inferred from the position of the words \sphinxstyleemphasis{to} and \sphinxstyleemphasis{from}.
\end{quote}

Optinal arguments:
\begin{quote}

\sphinxstyleemphasis{scale} : this keyword activates a relative scaling of the pattern of column densities
from the input ion (\sphinxstyleemphasis{ion1}) to the destination ion (\sphinxstyleemphasis{ion2}).
The keyword takes two arguments:
\begin{quote}

\sphinxstyleemphasis{logN} : the desired column density for the reference component
\sphinxstyleemphasis{ref\_comp} : the number of the reference component (starting from 0).
\end{quote}

Note: The default scaling is set to Solar relative abundances for the two elements.

\sphinxstyleemphasis{tie\_z} : If \sphinxstyleemphasis{True}, all redshifts for \sphinxstyleemphasis{ion2} will be tied to those of \sphinxstyleemphasis{ion1}.
Default is \sphinxstyleemphasis{True}.

\sphinxstyleemphasis{tie\_b} : If \sphinxstyleemphasis{True}, all \sphinxstyleemphasis{b}-parameters for \sphinxstyleemphasis{ion2} will be tied to those of \sphinxstyleemphasis{ion1}.
Default is \sphinxstyleemphasis{True}.
\end{quote}

\begin{sphinxShadowBox}
\sphinxstyletopictitle{Example}
\begin{description}
\item[{\sphinxcode{\sphinxupquote{copy components from FeII to SiII  scale 15.3  1}}}] \leavevmode
This will copy the component structure defined for FeII to SiII
and the logarithm of the column density of the 2nd component will be set to 15.3
while keeping the relative abundance pattern as defined for FeII.

\item[{\sphinxcode{\sphinxupquote{copy components to CII from FeII  tie\_b=False}}}] \leavevmode
This will copy components already defined for FeII to CII,
however, the broadening parameters are not fixed to those of FeII.
The initial value for log(N) for CII will be set using the Solar
relative abundance of carbon and iron.

\end{description}
\end{sphinxShadowBox}

delete component  number  {[}from{]} ion

number  gives the number of the component to delete (starting from 0).
ion             gives the ion from which to delete the given component.
\begin{quote}

Note: the keyword ‘from’ before the ion is optional.
\end{quote}

This function is useful for removing components that were defined using a “copy component” statement, if not all components should be fitted. For regular components, the component can simply be commented out (using ‘\#’).

Ex:
Suppose that FeII has 5 components defined and the same component structure has been copied to ZnII, which is much weaker. Therefore, only 4 components can be constrained for ZnII. This would be defined as follows:
component FeII  2.0456  15.5  14.6
component FeII  2.0469  11.5  14.8
component FeII  2.0482  17.5  13.3
component FeII  2.0489  14.0  14.3
component FeII  2.0495  13.5  14.7

copy components from FeII to ZnII  scale 13.2  0
delete component 2 from ZnII

resolution  res  {[} line\_tag {]}

res             gives the spectral resolution in km/s
line\_tag        specifies for which line\_tag the resolution should be changed. Default is all.

This function allows the user to update the spectral resolution. If some lines are defined in different spectra (loaded by different the “data” statements”) their spectral resolution will be different. Therefore, the spectral resolution should be updated for the given lines independently.

Warning: changing the spectral resolution in the “data” statement will not update the spectral resolution, unless the dataset is overwritten.

metallicity  logNHI  err\_logNHI

logNHI          the logarithm of the column density of neutral hydrogen in units of cm-2.
err\_logNHI      the uncertainty on the logarithm of the column density of neutral hydrogen.

When this keyword is present, the best-fit total abundances for the defined ions in the dataset will be converted to metallicities for each ion, that is, the abundance ratio of the given ion to neutral hydrogen relative to Solar abundances from Asplund et al. (2009) is calculated.

nomask

When this keyword is present in the parameter file (except in the dataset\_name), no interactive spectral masking will be performed.

mask  line\_tags

This keyword specifies individual lines to mask interactively. Used together with nomask, it allows the user to only mask a given set of lines and not all lines in the dataset.

abundance

When this keword is present in the parameter file (except in the dataset\_name), the total abundances for each ion will be printed to the terminal output.

reset  {[} line\_tags {]}

When this keword is present in the parameter file, the data for each region will be reset to the raw input data. This is used to update the continuum fitting so the code uses the raw data instead of the already normalized data in the regions. Note: This does not clear the spectral mask!

C\_order = 1

This keyword indicates the max order of Chebyshev polynomials to include for the continuum model. The default is 1, i.e., a straight line fit. The continuum is automatically optimized together with the line fitting.
By giving a negative order, the code will ask to manually normalize the fitting regions using the specified norm\_method, see below.

norm\_method = \{ ‘linear’  or  ‘spline’ \}

The norm\_method specifies how to manually normalize the fitting regions. Before fitting, each region will pop up and instructions will be given to normalize the data.
For ‘linear’, the user must specify a continuum region on the left of the absorption line (by clicking on the left and right boundaries of this continuum region) and similarly on the right side of the absoption line. The continuum is fitted using a straight line fit.
For ‘spline’, the user can select a range of points which will be fitted with a spline in order to create a curved continuum model.

systemic = value

This keyword defines how to update the systemic redshift after fitting.
Possible input values: ‘auto’, ‘none’ or  {[}num, ‘ion’{]}

Default behaviour is ‘none’: The systemic redshift will not be updated after fitting and the given systemic redshift (z\_sys) will be used.

If systemic is set to ‘auto’ the systemic redshift will be set to the redshift of the strongest component. The element used to identify the strongest component will be selected automatically, priority will be given to elements in the following order: ’FeII’ or ‘SiII’. If none of these are present, the first line in the dataset will be used. Warning: This may result in unexpected behaviour!

By giving an integer number (num) and an ion (separated by a comma), the user can force the systemic redshift to be set to that given component of the given ion after the fit has converged. Note that the components are 0-indexed, i.e., the first component is 0. If num is set to -1 then the last component of the given ion is used.

Example:
systemic  2   FeII
\begin{quote}

this defines the systemic redshift as the 3rd component of FeII
\end{quote}
\begin{description}
\item[{systemic  -1  SiII}] \leavevmode
this defines the systemic redshift as the last component of SiII

\end{description}

clear mask

This command will clear all the spectral masks that have been defined.


\chapter{Interface}
\label{\detokenize{index:interface}}

\section{VoigtFit Interface}
\label{\detokenize{api:voigtfit-interface}}\label{\detokenize{api::doc}}

\subsection{class \sphinxstylestrong{DataSet}}
\label{\detokenize{api:class-dataset}}\index{DataSet (class in VoigtFit)}

\begin{fulllineitems}
\phantomsection\label{\detokenize{api:VoigtFit.DataSet}}\pysiglinewithargsret{\sphinxbfcode{\sphinxupquote{class }}\sphinxcode{\sphinxupquote{VoigtFit.}}\sphinxbfcode{\sphinxupquote{DataSet}}}{\emph{redshift}, \emph{name=''}}{}
Main class of the package \sphinxcode{\sphinxupquote{VoigtFit}}. The DataSet handles all the major parts of the fit.
Spectral data must be added using the {\hyperref[\detokenize{api:VoigtFit.DataSet.add_data}]{\sphinxcrossref{\sphinxcode{\sphinxupquote{add\_data}}}}} method.
Hereafter the absorption lines to be fitted are added to the DataSet using the
{\hyperref[\detokenize{api:VoigtFit.DataSet.add_line}]{\sphinxcrossref{\sphinxcode{\sphinxupquote{add\_line}}}}} or
{\hyperref[\detokenize{api:VoigtFit.DataSet.add_many_lines}]{\sphinxcrossref{\sphinxcode{\sphinxupquote{add\_many\_lines}}}}} methods.
Lastly, the components of each element is defined using the
{\hyperref[\detokenize{api:VoigtFit.DataSet.add_component}]{\sphinxcrossref{\sphinxcode{\sphinxupquote{add\_component}}}}} method.
When all lines and components have been defined, the DataSet must be prepared by
calling the {\hyperref[\detokenize{api:VoigtFit.DataSet.prepare_dataset}]{\sphinxcrossref{\sphinxcode{\sphinxupquote{prepare\_dataset}}}}}
method and subsequently, the lines can be fitted using
the {\hyperref[\detokenize{api:VoigtFit.DataSet.fit}]{\sphinxcrossref{\sphinxcode{\sphinxupquote{fit}}}}} method.
\paragraph{Attributes}
\begin{description}
\item[{redshift}] \leavevmode{[}float{]}
Systemic redshift of the absorption system.

\item[{name}] \leavevmode{[}str   {[}default = ‘’{]}{]}
The name of the DataSet, this will be used for saving the dataset to a file structure.

\item[{verbose}] \leavevmode{[}bool   {[}default = True{]}{]}
If \sphinxtitleref{False}, the printed information statements will be suppressed.

\item[{data}] \leavevmode{[}list(data\_chunks){]}
A list of \sphinxstyleemphasis{data chunks} defined for the dataset. A \sphinxstyleemphasis{data chunk} is
a dictionary with keys ‘wl’, ‘flux’, ‘error’, ‘res’, ‘norm’.
See {\hyperref[\detokenize{api:VoigtFit.DataSet.add_data}]{\sphinxcrossref{\sphinxcode{\sphinxupquote{DataSet.add\_data}}}}}.

\item[{lines}] \leavevmode{[}dict{]}
A dictionary holding pairs of defined (\sphinxstyleemphasis{line\_tag} : {\hyperref[\detokenize{api:dataset.Line}]{\sphinxcrossref{\sphinxcode{\sphinxupquote{dataset.Line}}}}})

\item[{all\_lines}] \leavevmode{[}list(str){]}
A list of all the defined \sphinxstyleemphasis{line tags} for easy look-up.

\item[{molecules}] \leavevmode{[}dict{]}
A dictionary holding a list of the defined molecular bands and Jmax
for each molecule:
\sphinxcode{\sphinxupquote{\{molecule* : {[}{[}band1, Jmax1{]}, {[}band2, Jmax2{]}, etc...{]}\}}}

\item[{regions}] \leavevmode{[}list({\hyperref[\detokenize{api:regions.Region}]{\sphinxcrossref{\sphinxcode{\sphinxupquote{regions.Region}}}}}){]}
A list of the fitting regions.

\item[{cheb\_order}] \leavevmode{[}int   {[}default = -1{]}{]}
The maximum order of Chebyshev polynomials to use for the continuum
fitting in each region. If negative, the Chebyshev polynomials will
not be included in the fit.

\item[{norm\_method}] \leavevmode{[}str   {[}default = ‘linear’{]}{]}
Default normalization method to use for interactive normalization
if Chebyshev polynomial fitting should not be used.

\item[{components}] \leavevmode{[}dict{]}
A dictionary of components for each \sphinxstyleemphasis{ion} defined:
(\sphinxstyleemphasis{ion} : {[}z, b, logN, options{]}). See {\hyperref[\detokenize{api:VoigtFit.DataSet.add_component}]{\sphinxcrossref{\sphinxcode{\sphinxupquote{DataSet.add\_component}}}}}.

\item[{velspan}] \leavevmode{[}float   {[}default = 500{]}{]}
The default velocity range to use for the definition
of fitting regions.

\item[{ready2fit}] \leavevmode{[}bool   {[}default = False{]}{]}
This attribute is checked before fitting the dataset. Only when
the attribute has been set to \sphinxtitleref{True} can the dataset be fitted.
This will be toggled after a successful run of
{\hyperref[\detokenize{api:VoigtFit.DataSet.prepare_dataset}]{\sphinxcrossref{\sphinxcode{\sphinxupquote{DataSet.prepare\_dataset}}}}}.

\item[{best\_fit}] \leavevmode{[}\sphinxhref{https://lmfit.github.io/lmfit-py/parameters.html}{lmfit.Parameters}   {[}default = None{]}{]}
Best-fit parameters from \sphinxhref{https://lmfit.github.io/lmfit-py/}{lmfit}.
This attribute will be \sphinxtitleref{None} until the dataset has been fitted.

\item[{pars}] \leavevmode{[}\sphinxhref{https://lmfit.github.io/lmfit-py/parameters.html}{lmfit.Parameters}   {[}default = None{]}{]}
Placeholder for the fit parameters initiated before the fit.
The parameters will be defined during the call to {\hyperref[\detokenize{api:VoigtFit.DataSet.prepare_dataset}]{\sphinxcrossref{\sphinxcode{\sphinxupquote{DataSet.prepare\_dataset}}}}} based on the defined components.

\end{description}
\paragraph{Methods}


\begin{savenotes}\sphinxatlongtablestart\begin{longtable}{p{0.5\linewidth}p{0.5\linewidth}}
\hline

\endfirsthead

\multicolumn{2}{c}%
{\makebox[0pt]{\sphinxtablecontinued{\tablename\ \thetable{} -- continued from previous page}}}\\
\hline

\endhead

\hline
\multicolumn{2}{r}{\makebox[0pt][r]{\sphinxtablecontinued{Continued on next page}}}\\
\endfoot

\endlastfoot

{\hyperref[\detokenize{api:VoigtFit.DataSet.activate_all}]{\sphinxcrossref{\sphinxcode{\sphinxupquote{activate\_all}}}}}()
&
Activate all lines defined in the DataSet.
\\
\hline
{\hyperref[\detokenize{api:VoigtFit.DataSet.activate_fine_lines}]{\sphinxcrossref{\sphinxcode{\sphinxupquote{activate\_fine\_lines}}}}}(line\_tag{[}, levels{]})
&
Activate all lines associated to a given fine-structure complex.
\\
\hline
{\hyperref[\detokenize{api:VoigtFit.DataSet.activate_line}]{\sphinxcrossref{\sphinxcode{\sphinxupquote{activate\_line}}}}}(line\_tag)
&
Activate a given line defined by its \sphinxtitleref{line\_tag}
\\
\hline
{\hyperref[\detokenize{api:VoigtFit.DataSet.activate_molecule}]{\sphinxcrossref{\sphinxcode{\sphinxupquote{activate\_molecule}}}}}(molecule, band)
&
Activate all lines for the given band of the given molecule.
\\
\hline
{\hyperref[\detokenize{api:VoigtFit.DataSet.add_component}]{\sphinxcrossref{\sphinxcode{\sphinxupquote{add\_component}}}}}(ion, z, b, logN{[}, var\_z, …{]})
&
Add component for a given ion.
\\
\hline
{\hyperref[\detokenize{api:VoigtFit.DataSet.add_component_velocity}]{\sphinxcrossref{\sphinxcode{\sphinxupquote{add\_component\_velocity}}}}}(ion, v, b, logN{[}, …{]})
&
Same as for {\hyperref[\detokenize{api:VoigtFit.DataSet.add_component}]{\sphinxcrossref{\sphinxcode{\sphinxupquote{add\_component}}}}} but input is given as relative velocity instead of redshift.
\\
\hline
{\hyperref[\detokenize{api:VoigtFit.DataSet.add_data}]{\sphinxcrossref{\sphinxcode{\sphinxupquote{add\_data}}}}}(wl, flux, res{[}, err, normalized, mask{]})
&
Add spectral data to the DataSet.
\\
\hline
{\hyperref[\detokenize{api:VoigtFit.DataSet.add_fine_lines}]{\sphinxcrossref{\sphinxcode{\sphinxupquote{add\_fine\_lines}}}}}(line\_tag{[}, levels, full\_label{]})
&
Add fine-structure line complexes by providing only the main transition.
\\
\hline
{\hyperref[\detokenize{api:VoigtFit.DataSet.add_line}]{\sphinxcrossref{\sphinxcode{\sphinxupquote{add\_line}}}}}(line\_tag{[}, velspan, active{]})
&
Add an absorption line to the DataSet.
\\
\hline
{\hyperref[\detokenize{api:VoigtFit.DataSet.add_lines}]{\sphinxcrossref{\sphinxcode{\sphinxupquote{add\_lines}}}}}(line\_tags{[}, velspan{]})
&
Alias for \sphinxtitleref{self.add\_many\_lines}.
\\
\hline
{\hyperref[\detokenize{api:VoigtFit.DataSet.add_many_lines}]{\sphinxcrossref{\sphinxcode{\sphinxupquote{add\_many\_lines}}}}}(tags{[}, velspan{]})
&
Add many lines at once.
\\
\hline
{\hyperref[\detokenize{api:VoigtFit.DataSet.add_molecule}]{\sphinxcrossref{\sphinxcode{\sphinxupquote{add\_molecule}}}}}(molecule, band{[}, Jmax, …{]})
&
Add molecular lines for a given band, e.g., “AX(0-0)” of CO.
\\
\hline
{\hyperref[\detokenize{api:VoigtFit.DataSet.all_active_lines}]{\sphinxcrossref{\sphinxcode{\sphinxupquote{all\_active\_lines}}}}}()
&
Returns a list of all the active lines defined by their \sphinxtitleref{line\_tag}.
\\
\hline
{\hyperref[\detokenize{api:VoigtFit.DataSet.clear_mask}]{\sphinxcrossref{\sphinxcode{\sphinxupquote{clear\_mask}}}}}(line\_tag{[}, idx{]})
&
Clear the mask for the {\hyperref[\detokenize{api:regions.Region}]{\sphinxcrossref{\sphinxcode{\sphinxupquote{Region}}}}} containing the given \sphinxtitleref{line\_tag}.
\\
\hline
{\hyperref[\detokenize{api:VoigtFit.DataSet.copy_components}]{\sphinxcrossref{\sphinxcode{\sphinxupquote{copy\_components}}}}}({[}to\_ion, from\_ion, logN, …{]})
&
Copy velocity structure to \sphinxtitleref{ion} from another \sphinxtitleref{ion}.
\\
\hline
{\hyperref[\detokenize{api:VoigtFit.DataSet.deactivate_all}]{\sphinxcrossref{\sphinxcode{\sphinxupquote{deactivate\_all}}}}}()
&
Deactivate all lines defined in the DataSet.
\\
\hline
{\hyperref[\detokenize{api:VoigtFit.DataSet.deactivate_fine_lines}]{\sphinxcrossref{\sphinxcode{\sphinxupquote{deactivate\_fine\_lines}}}}}(line\_tag{[}, levels{]})
&
Deactivate all lines associated to a given fine-structure complex.
\\
\hline
{\hyperref[\detokenize{api:VoigtFit.DataSet.deactivate_line}]{\sphinxcrossref{\sphinxcode{\sphinxupquote{deactivate\_line}}}}}(line\_tag)
&
Deactivate a given line defined by its \sphinxtitleref{line\_tag}.
\\
\hline
{\hyperref[\detokenize{api:VoigtFit.DataSet.deactivate_molecule}]{\sphinxcrossref{\sphinxcode{\sphinxupquote{deactivate\_molecule}}}}}(molecule, band)
&
Deactivate all lines for the given band of the given molecule.
\\
\hline
{\hyperref[\detokenize{api:VoigtFit.DataSet.delete_component}]{\sphinxcrossref{\sphinxcode{\sphinxupquote{delete\_component}}}}}(ion, index)
&
Remove component of the given \sphinxtitleref{ion} with the given \sphinxtitleref{index}.
\\
\hline
{\hyperref[\detokenize{api:VoigtFit.DataSet.find_ion}]{\sphinxcrossref{\sphinxcode{\sphinxupquote{find\_ion}}}}}(ion)
&
Return a list of all line tags for a given ion.
\\
\hline
{\hyperref[\detokenize{api:VoigtFit.DataSet.find_line}]{\sphinxcrossref{\sphinxcode{\sphinxupquote{find\_line}}}}}(line\_tag)
&
Look up the fitting {\hyperref[\detokenize{api:regions.Region}]{\sphinxcrossref{\sphinxcode{\sphinxupquote{Region}}}}} for a given \sphinxstyleemphasis{line tag}.
\\
\hline
{\hyperref[\detokenize{api:VoigtFit.DataSet.fit}]{\sphinxcrossref{\sphinxcode{\sphinxupquote{fit}}}}}({[}rebin, verbose, plot{]})
&
Fit the absorption lines using chi-square minimization.
\\
\hline
{\hyperref[\detokenize{api:VoigtFit.DataSet.fix_structure}]{\sphinxcrossref{\sphinxcode{\sphinxupquote{fix\_structure}}}}}({[}ion{]})
&
Fix the velocity structure, that is, the redshifts and the b-parameters.
\\
\hline
{\hyperref[\detokenize{api:VoigtFit.DataSet.free_structure}]{\sphinxcrossref{\sphinxcode{\sphinxupquote{free\_structure}}}}}({[}ion{]})
&
Free the velocity structure, that is, the redshifts and the b-parameters.
\\
\hline
{\hyperref[\detokenize{api:VoigtFit.DataSet.get_lines_for_ion}]{\sphinxcrossref{\sphinxcode{\sphinxupquote{get\_lines\_for\_ion}}}}}(ion)
&
Return a list of {\hyperref[\detokenize{api:dataset.Line}]{\sphinxcrossref{\sphinxcode{\sphinxupquote{Line}}}}} objects corresponding to the given \sphinxstyleemphasis{ion}.
\\
\hline
{\hyperref[\detokenize{api:VoigtFit.DataSet.get_name}]{\sphinxcrossref{\sphinxcode{\sphinxupquote{get\_name}}}}}()
&
Returns the name of the DataSet.
\\
\hline
{\hyperref[\detokenize{api:VoigtFit.DataSet.get_resolution}]{\sphinxcrossref{\sphinxcode{\sphinxupquote{get\_resolution}}}}}(line\_tag{[}, verbose{]})
&
Return the spectral resolution for the fitting {\hyperref[\detokenize{api:regions.Region}]{\sphinxcrossref{\sphinxcode{\sphinxupquote{Region}}}}} where the line with the given \sphinxtitleref{line\_tag} is defined.
\\
\hline
{\hyperref[\detokenize{api:VoigtFit.DataSet.has_ion}]{\sphinxcrossref{\sphinxcode{\sphinxupquote{has\_ion}}}}}(ion{[}, active\_only{]})
&
Return True if the dataset has lines defined for the given ion.
\\
\hline
{\hyperref[\detokenize{api:VoigtFit.DataSet.has_line}]{\sphinxcrossref{\sphinxcode{\sphinxupquote{has\_line}}}}}(line\_tag{[}, active\_only{]})
&
Return True if the given line is defined.
\\
\hline
{\hyperref[\detokenize{api:VoigtFit.DataSet.interactive_components}]{\sphinxcrossref{\sphinxcode{\sphinxupquote{interactive\_components}}}}}(line\_tag)
&
Define components interactively for a given ion.
\\
\hline
{\hyperref[\detokenize{api:VoigtFit.DataSet.load_components_from_file}]{\sphinxcrossref{\sphinxcode{\sphinxupquote{load\_components\_from\_file}}}}}(fname)
&
Load best-fit parameters from an output file \sphinxtitleref{fname}.
\\
\hline
{\hyperref[\detokenize{api:VoigtFit.DataSet.mask_line}]{\sphinxcrossref{\sphinxcode{\sphinxupquote{mask\_line}}}}}(line\_tag{[}, reset, mask, telluric{]})
&
Define exclusion masks for the fitting region of a given line.
\\
\hline
{\hyperref[\detokenize{api:VoigtFit.DataSet.normalize_line}]{\sphinxcrossref{\sphinxcode{\sphinxupquote{normalize\_line}}}}}(line\_tag{[}, norm\_method{]})
&
Normalize or re-normalize a given line
\\
\hline
{\hyperref[\detokenize{api:VoigtFit.DataSet.plot_fit}]{\sphinxcrossref{\sphinxcode{\sphinxupquote{plot\_fit}}}}}({[}rebin, fontsize, xmin, xmax, …{]})
&
Plot \sphinxstyleemphasis{all} the absorption lines and the best-fit profiles.
\\
\hline
{\hyperref[\detokenize{api:VoigtFit.DataSet.plot_line}]{\sphinxcrossref{\sphinxcode{\sphinxupquote{plot\_line}}}}}(line\_tag{[}, index, plot\_fit, loc, …{]})
&
Plot a single fitting {\hyperref[\detokenize{api:regions.Region}]{\sphinxcrossref{\sphinxcode{\sphinxupquote{Region}}}}} containing the line corresponding to the given \sphinxtitleref{line\_tag}.
\\
\hline
{\hyperref[\detokenize{api:VoigtFit.DataSet.prepare_dataset}]{\sphinxcrossref{\sphinxcode{\sphinxupquote{prepare\_dataset}}}}}({[}norm, mask, verbose, …{]})
&
Prepare the data for fitting.
\\
\hline
{\hyperref[\detokenize{api:VoigtFit.DataSet.print_cont_parameters}]{\sphinxcrossref{\sphinxcode{\sphinxupquote{print\_cont\_parameters}}}}}()
&
Print Chebyshev coefficients for the continuum fit.
\\
\hline
{\hyperref[\detokenize{api:VoigtFit.DataSet.print_metallicity}]{\sphinxcrossref{\sphinxcode{\sphinxupquote{print\_metallicity}}}}}(logNHI{[}, err{]})
&
Print the total column densities for each element relative to HI in Solar units.
\\
\hline
{\hyperref[\detokenize{api:VoigtFit.DataSet.print_results}]{\sphinxcrossref{\sphinxcode{\sphinxupquote{print\_results}}}}}({[}velocity, elements, systemic{]})
&
Print the best fit parameters.
\\
\hline
{\hyperref[\detokenize{api:VoigtFit.DataSet.print_total}]{\sphinxcrossref{\sphinxcode{\sphinxupquote{print\_total}}}}}()
&
Print the total column densities of all components.
\\
\hline
\sphinxcode{\sphinxupquote{remove\_all\_lines}}()
&

\\
\hline
{\hyperref[\detokenize{api:VoigtFit.DataSet.remove_fine_lines}]{\sphinxcrossref{\sphinxcode{\sphinxupquote{remove\_fine\_lines}}}}}(line\_tag{[}, levels{]})
&
Remove lines associated to a given fine-structure complex.
\\
\hline
{\hyperref[\detokenize{api:VoigtFit.DataSet.remove_line}]{\sphinxcrossref{\sphinxcode{\sphinxupquote{remove\_line}}}}}(line\_tag)
&
Remove an absorption line from the DataSet.
\\
\hline
{\hyperref[\detokenize{api:VoigtFit.DataSet.remove_molecule}]{\sphinxcrossref{\sphinxcode{\sphinxupquote{remove\_molecule}}}}}(molecule, band)
&
Remove all lines for the given band of the given molecule.
\\
\hline
{\hyperref[\detokenize{api:VoigtFit.DataSet.reset_all_regions}]{\sphinxcrossref{\sphinxcode{\sphinxupquote{reset\_all\_regions}}}}}()
&
Reset the data in all {\hyperref[\detokenize{api:regions.Region}]{\sphinxcrossref{\sphinxcode{\sphinxupquote{Regions}}}}} defined in the DataSet to use the raw input data.
\\
\hline
{\hyperref[\detokenize{api:VoigtFit.DataSet.reset_components}]{\sphinxcrossref{\sphinxcode{\sphinxupquote{reset\_components}}}}}({[}ion{]})
&
Reset component structure for a given ion.
\\
\hline
{\hyperref[\detokenize{api:VoigtFit.DataSet.reset_region}]{\sphinxcrossref{\sphinxcode{\sphinxupquote{reset\_region}}}}}(reg)
&
Reset the data in a given {\hyperref[\detokenize{api:regions.Region}]{\sphinxcrossref{\sphinxcode{\sphinxupquote{regions.Region}}}}} to use the raw input data.
\\
\hline
{\hyperref[\detokenize{api:VoigtFit.DataSet.save}]{\sphinxcrossref{\sphinxcode{\sphinxupquote{save}}}}}({[}filename, verbose{]})
&
Save the DataSet to file using the HDF5 format.
\\
\hline
{\hyperref[\detokenize{api:VoigtFit.DataSet.save_cont_parameters_to_file}]{\sphinxcrossref{\sphinxcode{\sphinxupquote{save\_cont\_parameters\_to\_file}}}}}(filename)
&
Save the best-fit continuum parameters to ASCII table output.
\\
\hline
{\hyperref[\detokenize{api:VoigtFit.DataSet.save_fit_regions}]{\sphinxcrossref{\sphinxcode{\sphinxupquote{save\_fit\_regions}}}}}({[}filename, individual, path{]})
&
Save the fitting regions to ASCII table output.
\\
\hline
{\hyperref[\detokenize{api:VoigtFit.DataSet.save_parameters}]{\sphinxcrossref{\sphinxcode{\sphinxupquote{save\_parameters}}}}}(filename)
&
Save the best-fit parameters to ASCII table output.
\\
\hline
{\hyperref[\detokenize{api:VoigtFit.DataSet.set_name}]{\sphinxcrossref{\sphinxcode{\sphinxupquote{set\_name}}}}}(name)
&
Set the name of the DataSet.
\\
\hline
{\hyperref[\detokenize{api:VoigtFit.DataSet.set_resolution}]{\sphinxcrossref{\sphinxcode{\sphinxupquote{set\_resolution}}}}}(res{[}, line\_tag, verbose{]})
&
Set the spectral resolution in km/s for the {\hyperref[\detokenize{api:regions.Region}]{\sphinxcrossref{\sphinxcode{\sphinxupquote{Region}}}}} containing the line with the given \sphinxtitleref{line\_tag}.
\\
\hline
{\hyperref[\detokenize{api:VoigtFit.DataSet.set_systemic_redshift}]{\sphinxcrossref{\sphinxcode{\sphinxupquote{set\_systemic\_redshift}}}}}(z\_sys)
&
Update the systemic redshift of the dataset
\\
\hline
{\hyperref[\detokenize{api:VoigtFit.DataSet.sum_components}]{\sphinxcrossref{\sphinxcode{\sphinxupquote{sum\_components}}}}}(ions, components)
&
Calculate the total column density for the given \sphinxtitleref{components} of the given \sphinxtitleref{ion}.
\\
\hline
{\hyperref[\detokenize{api:VoigtFit.DataSet.velocity_plot}]{\sphinxcrossref{\sphinxcode{\sphinxupquote{velocity\_plot}}}}}(**kwargs)
&
Create a velocity plot, showing all the fitting regions defined, in order to compare different lines and to identify blends and contamination.
\\
\hline
\end{longtable}\sphinxatlongtableend\end{savenotes}
\index{activate\_all() (VoigtFit.DataSet method)}

\begin{fulllineitems}
\phantomsection\label{\detokenize{api:VoigtFit.DataSet.activate_all}}\pysiglinewithargsret{\sphinxbfcode{\sphinxupquote{activate\_all}}}{}{}
Activate all lines defined in the DataSet.

\end{fulllineitems}

\index{activate\_fine\_lines() (VoigtFit.DataSet method)}

\begin{fulllineitems}
\phantomsection\label{\detokenize{api:VoigtFit.DataSet.activate_fine_lines}}\pysiglinewithargsret{\sphinxbfcode{\sphinxupquote{activate\_fine\_lines}}}{\emph{line\_tag}, \emph{levels=None}}{}
Activate all lines associated to a given fine-structure complex.
\begin{quote}\begin{description}
\item[{Parameters}] \leavevmode
\sphinxstylestrong{line\_tag} : str
\begin{quote}

The line tag of the ground state transition to activate.
\end{quote}

\sphinxstylestrong{levels} : str, list(str)   {[}default = None{]}
\begin{quote}

The levels of the fine-structure complexes to activate,
with the string “a” referring to the first excited level,
“b” is the second, etc…
Several levels can be given at once as a list: {[}‘a’, ‘b’{]}
or as a concatenated string: ‘abc’.
By default, all levels are included.
\end{quote}

\end{description}\end{quote}

\end{fulllineitems}

\index{activate\_line() (VoigtFit.DataSet method)}

\begin{fulllineitems}
\phantomsection\label{\detokenize{api:VoigtFit.DataSet.activate_line}}\pysiglinewithargsret{\sphinxbfcode{\sphinxupquote{activate\_line}}}{\emph{line\_tag}}{}
Activate a given line defined by its \sphinxtitleref{line\_tag}

\end{fulllineitems}

\index{activate\_molecule() (VoigtFit.DataSet method)}

\begin{fulllineitems}
\phantomsection\label{\detokenize{api:VoigtFit.DataSet.activate_molecule}}\pysiglinewithargsret{\sphinxbfcode{\sphinxupquote{activate\_molecule}}}{\emph{molecule}, \emph{band}}{}
Activate all lines for the given band of the given molecule.
\begin{itemize}
\item {} 
Ex: \sphinxcode{\sphinxupquote{activate\_molecule('CO', 'AX(0-0)')}}

\end{itemize}

\end{fulllineitems}

\index{add\_component() (VoigtFit.DataSet method)}

\begin{fulllineitems}
\phantomsection\label{\detokenize{api:VoigtFit.DataSet.add_component}}\pysiglinewithargsret{\sphinxbfcode{\sphinxupquote{add\_component}}}{\emph{ion}, \emph{z}, \emph{b}, \emph{logN}, \emph{var\_z=True}, \emph{var\_b=True}, \emph{var\_N=True}, \emph{tie\_z=None}, \emph{tie\_b=None}, \emph{tie\_N=None}}{}
Add component for a given ion. Each component defined will be used for all transitions
defined for a given ion.
\begin{quote}\begin{description}
\item[{Parameters}] \leavevmode
\sphinxstylestrong{ion} : str
\begin{quote}

The ion for which to define a component: e.g., “FeII”, “HI”, “CIa”, etc.
\end{quote}

\sphinxstylestrong{z} : float
\begin{quote}

The redshift of the component.
\end{quote}

\sphinxstylestrong{b} : float
\begin{quote}

The effective broadening parameter for the component in km/s.
This parameter is constrained to be in the interval {[}0 - 1000{]} km/s.
\end{quote}

\sphinxstylestrong{logN} : float
\begin{quote}

The 10-base logarithm of the column density of the component.
The column density is expected in cm\textasciicircum{}-2.
\end{quote}

\sphinxstylestrong{var\_z} : bool   {[}default = True{]}
\begin{quote}

If \sphinxtitleref{False}, the redshift of the component will be kept fixed.
\end{quote}

\sphinxstylestrong{var\_b} : bool   {[}default = True{]}
\begin{quote}

If \sphinxtitleref{False}, the b-parameter of the component will be kept fixed.
\end{quote}

\sphinxstylestrong{var\_N} : bool   {[}default = True{]}
\begin{quote}

If \sphinxtitleref{False}, the column density of the component will be kept fixed.
\end{quote}

\sphinxstylestrong{tie\_z, tie\_b, tie\_N} : str   {[}default = None{]}
\begin{quote}

Parameter constraints for the different variables.
\end{quote}

\end{description}\end{quote}
\paragraph{Notes}

The ties are defined relative to the parameter names. The naming is as follows:
The redshift of the first component of FeII is called “z0\_FeII”,
the logN of the second component of SiII is called “logN1\_SiII”.
For more information about parameter ties, see the documentation for \sphinxhref{https://lmfit.github.io/lmfit-py/}{lmfit}.

\end{fulllineitems}

\index{add\_component\_velocity() (VoigtFit.DataSet method)}

\begin{fulllineitems}
\phantomsection\label{\detokenize{api:VoigtFit.DataSet.add_component_velocity}}\pysiglinewithargsret{\sphinxbfcode{\sphinxupquote{add\_component\_velocity}}}{\emph{ion}, \emph{v}, \emph{b}, \emph{logN}, \emph{var\_z=True}, \emph{var\_b=True}, \emph{var\_N=True}, \emph{tie\_z=None}, \emph{tie\_b=None}, \emph{tie\_N=None}}{}
Same as for {\hyperref[\detokenize{api:VoigtFit.DataSet.add_component}]{\sphinxcrossref{\sphinxcode{\sphinxupquote{add\_component}}}}}
but input is given as relative velocity instead of redshift.

\end{fulllineitems}

\index{add\_data() (VoigtFit.DataSet method)}

\begin{fulllineitems}
\phantomsection\label{\detokenize{api:VoigtFit.DataSet.add_data}}\pysiglinewithargsret{\sphinxbfcode{\sphinxupquote{add\_data}}}{\emph{wl}, \emph{flux}, \emph{res}, \emph{err=None}, \emph{normalized=False}, \emph{mask=None}}{}
Add spectral data to the DataSet. This will be used to define fitting regions.
\begin{quote}\begin{description}
\item[{Parameters}] \leavevmode
\sphinxstylestrong{wl} : ndarray, shape (n)
\begin{quote}

Input vacuum wavelength array in Ångstrøm.
\end{quote}

\sphinxstylestrong{flux} : ndarray, shape (n)
\begin{quote}

Input flux array, should be same length as wl
\end{quote}

\sphinxstylestrong{res} : float
\begin{quote}

Spectral resolution in km/s  (c/R)
\end{quote}

\sphinxstylestrong{err} : ndarray, shape (n)   {[}default = None{]}
\begin{quote}

Error array, should be same length as wl
If \sphinxtitleref{None} is given, an uncertainty of 1 is given to all pixels.
\end{quote}

\sphinxstylestrong{normalized} : bool   {[}default = False{]}
\begin{quote}

If the input spectrum is normalized this should be given as True
in order to skip normalization steps.
\end{quote}

\sphinxstylestrong{mask} : array, shape (n)
\begin{quote}

Boolean/int array defining the fitting regions.
Only pixels with mask=True/1 will be included in the fit.
\end{quote}

\end{description}\end{quote}

\end{fulllineitems}

\index{add\_fine\_lines() (VoigtFit.DataSet method)}

\begin{fulllineitems}
\phantomsection\label{\detokenize{api:VoigtFit.DataSet.add_fine_lines}}\pysiglinewithargsret{\sphinxbfcode{\sphinxupquote{add\_fine\_lines}}}{\emph{line\_tag}, \emph{levels=None}, \emph{full\_label=False}}{}
Add fine-structure line complexes by providing only the main transition.
This function is mainly useful for the CI complexes, where the many lines are closely
located and often blended.
\begin{quote}\begin{description}
\item[{Parameters}] \leavevmode
\sphinxstylestrong{line\_tag} : str
\begin{quote}

Line tag for the ground state transition, e.g., “CI\_1656”
\end{quote}

\sphinxstylestrong{levels} : str, list(str)    {[}default = None{]}
\begin{quote}

The levels of the fine-structure complexes to add, starting with “a” referring
to the first excited level, “b” is the second, etc..
Several levels can be given at once: {[}‘a’, ‘b’{]}
By default, all levels are included.
\end{quote}

\sphinxstylestrong{full\_label} : bool   {[}default = False{]}
\begin{quote}

If \sphinxtitleref{True}, the label will be translated to the full quantum mechanical description
of the state.
\end{quote}

\end{description}\end{quote}

\end{fulllineitems}

\index{add\_line() (VoigtFit.DataSet method)}

\begin{fulllineitems}
\phantomsection\label{\detokenize{api:VoigtFit.DataSet.add_line}}\pysiglinewithargsret{\sphinxbfcode{\sphinxupquote{add\_line}}}{\emph{line\_tag}, \emph{velspan=None}, \emph{active=True}}{}
Add an absorption line to the DataSet.
\begin{quote}\begin{description}
\item[{Parameters}] \leavevmode
\sphinxstylestrong{line\_tag} : str
\begin{quote}

The line tag for the transition which should be defined.
Ex: “FeII\_2374”
\end{quote}

\sphinxstylestrong{velspan} : float   {[}default = None{]}
\begin{quote}

The velocity span around the line center, which will be included
in the fit. If \sphinxtitleref{None} is given, use the default \sphinxtitleref{self.velspan}
defined (default = 500 km/s).
\end{quote}

\sphinxstylestrong{active} : bool   {[}default = True{]}
\begin{quote}

Set the \sphinxcode{\sphinxupquote{Line}} as active
(i.e., included in the fit).
\end{quote}

\end{description}\end{quote}
\paragraph{Notes}

This will initiate a \sphinxcode{\sphinxupquote{Line}} class
with the atomic data for the transition, as well as creating a
fitting {\hyperref[\detokenize{api:regions.Region}]{\sphinxcrossref{\sphinxcode{\sphinxupquote{Region}}}}} containing the data cutout
around the line center.

\end{fulllineitems}

\index{add\_lines() (VoigtFit.DataSet method)}

\begin{fulllineitems}
\phantomsection\label{\detokenize{api:VoigtFit.DataSet.add_lines}}\pysiglinewithargsret{\sphinxbfcode{\sphinxupquote{add\_lines}}}{\emph{line\_tags}, \emph{velspan=None}}{}
Alias for \sphinxtitleref{self.add\_many\_lines}.

\end{fulllineitems}

\index{add\_many\_lines() (VoigtFit.DataSet method)}

\begin{fulllineitems}
\phantomsection\label{\detokenize{api:VoigtFit.DataSet.add_many_lines}}\pysiglinewithargsret{\sphinxbfcode{\sphinxupquote{add\_many\_lines}}}{\emph{tags}, \emph{velspan=None}}{}
Add many lines at once.
\begin{quote}\begin{description}
\item[{Parameters}] \leavevmode
\sphinxstylestrong{tags} : list(str)
\begin{quote}

A list of line tags for the transitions that should be added.
\end{quote}

\sphinxstylestrong{velspan} : float   {[}default = None{]}
\begin{quote}

The velocity span around the line center, which will be included
in the fit. If \sphinxtitleref{None} is given, use the default
\sphinxcode{\sphinxupquote{velspan}} defined (500 km/s).
\end{quote}

\end{description}\end{quote}

\end{fulllineitems}

\index{add\_molecule() (VoigtFit.DataSet method)}

\begin{fulllineitems}
\phantomsection\label{\detokenize{api:VoigtFit.DataSet.add_molecule}}\pysiglinewithargsret{\sphinxbfcode{\sphinxupquote{add\_molecule}}}{\emph{molecule}, \emph{band}, \emph{Jmax=0}, \emph{velspan=None}, \emph{full\_label=False}}{}
Add molecular lines for a given band, e.g., “AX(0-0)” of CO.
\begin{quote}\begin{description}
\item[{Parameters}] \leavevmode
\sphinxstylestrong{molecule} : str
\begin{quote}

The molecular identifier, e.g., ‘CO’, ‘H2’
\end{quote}

\sphinxstylestrong{band} : str
\begin{quote}

The vibrational band of the molecule, e.g., for CO: “AX(0-0)”
These bands are defined in \sphinxcode{\sphinxupquote{molecules}}.
\end{quote}

\sphinxstylestrong{Jmax} : int   {[}default = 0{]}
\begin{quote}

The maximal rotational level to include. All levels up to and including \sphinxtitleref{J}
will be included.
\end{quote}

\sphinxstylestrong{velspan} : float   {[}default = None{]}
\begin{quote}

The velocity span around the line center, which will be included in the fit.
If \sphinxtitleref{None} is given, use the default \sphinxcode{\sphinxupquote{velspan}}
defined (500 km/s).
\end{quote}

\sphinxstylestrong{full\_label} : bool   {[}default = False{]}
\begin{quote}

If \sphinxtitleref{True}, the label will be translated to the full quantum
mechanical description of the state.
\end{quote}

\end{description}\end{quote}

\end{fulllineitems}

\index{all\_active\_lines() (VoigtFit.DataSet method)}

\begin{fulllineitems}
\phantomsection\label{\detokenize{api:VoigtFit.DataSet.all_active_lines}}\pysiglinewithargsret{\sphinxbfcode{\sphinxupquote{all\_active\_lines}}}{}{}
Returns a list of all the active lines defined by their \sphinxtitleref{line\_tag}.

\end{fulllineitems}

\index{clear\_mask() (VoigtFit.DataSet method)}

\begin{fulllineitems}
\phantomsection\label{\detokenize{api:VoigtFit.DataSet.clear_mask}}\pysiglinewithargsret{\sphinxbfcode{\sphinxupquote{clear\_mask}}}{\emph{line\_tag}, \emph{idx=None}}{}
Clear the mask for the {\hyperref[\detokenize{api:regions.Region}]{\sphinxcrossref{\sphinxcode{\sphinxupquote{Region}}}}}
containing the given \sphinxtitleref{line\_tag}.
If more regions are defined for the same line (when fitting multiple spectra),
the given region can be specified by passing an index \sphinxtitleref{idx}.

\end{fulllineitems}

\index{copy\_components() (VoigtFit.DataSet method)}

\begin{fulllineitems}
\phantomsection\label{\detokenize{api:VoigtFit.DataSet.copy_components}}\pysiglinewithargsret{\sphinxbfcode{\sphinxupquote{copy\_components}}}{\emph{to\_ion=''}, \emph{from\_ion=''}, \emph{logN=0}, \emph{ref\_comp=None}, \emph{tie\_z=True}, \emph{tie\_b=True}}{}
Copy velocity structure to \sphinxtitleref{ion} from another \sphinxtitleref{ion}.
\begin{quote}\begin{description}
\item[{Parameters}] \leavevmode
\sphinxstylestrong{to\_ion} : str
\begin{quote}

The new ion to define.
\end{quote}

\sphinxstylestrong{from\_ion} : str
\begin{quote}

The base ion which will be used as reference.
\end{quote}

\sphinxstylestrong{logN} : float
\begin{quote}

If logN is given, the starting guess is defined from this value
following the pattern of the components defined for \sphinxtitleref{anchor} relative to the
\sphinxtitleref{ref\_comp} (default: the first component).
\end{quote}

\sphinxstylestrong{ref\_comp} : int
\begin{quote}

The reference component to which logN will be scaled.
\end{quote}

\sphinxstylestrong{tie\_z} : bool   {[}default = True{]}
\begin{quote}

If \sphinxtitleref{True}, the redshifts for all components of the two ions will be tied together.
\end{quote}

\sphinxstylestrong{tie\_b} : bool   {[}default = True{]}
\begin{quote}

If \sphinxtitleref{True}, the b-parameters for all components of the two ions will be tied together.
\end{quote}

\end{description}\end{quote}

\end{fulllineitems}

\index{deactivate\_all() (VoigtFit.DataSet method)}

\begin{fulllineitems}
\phantomsection\label{\detokenize{api:VoigtFit.DataSet.deactivate_all}}\pysiglinewithargsret{\sphinxbfcode{\sphinxupquote{deactivate\_all}}}{}{}
Deactivate all lines defined in the DataSet. This will not remove the lines.

\end{fulllineitems}

\index{deactivate\_fine\_lines() (VoigtFit.DataSet method)}

\begin{fulllineitems}
\phantomsection\label{\detokenize{api:VoigtFit.DataSet.deactivate_fine_lines}}\pysiglinewithargsret{\sphinxbfcode{\sphinxupquote{deactivate\_fine\_lines}}}{\emph{line\_tag}, \emph{levels=None}}{}
Deactivate all lines associated to a given fine-structure complex.
\begin{quote}\begin{description}
\item[{Parameters}] \leavevmode
\sphinxstylestrong{line\_tag} : str
\begin{quote}

The line tag of the ground state transition to deactivate.
\end{quote}

\sphinxstylestrong{levels} : str, list(str)   {[}default = None{]}
\begin{quote}

The levels of the fine-structure complexes to deactivate,
with the string “a” referring to the first excited level,
“b” is the second, etc…
Several levels can be given at once as a list: {[}‘a’, ‘b’{]}
or as a concatenated string: ‘abc’.
By default, all levels are included.
\end{quote}

\end{description}\end{quote}

\end{fulllineitems}

\index{deactivate\_line() (VoigtFit.DataSet method)}

\begin{fulllineitems}
\phantomsection\label{\detokenize{api:VoigtFit.DataSet.deactivate_line}}\pysiglinewithargsret{\sphinxbfcode{\sphinxupquote{deactivate\_line}}}{\emph{line\_tag}}{}
Deactivate a given line defined by its \sphinxtitleref{line\_tag}.
This will exclude the line during the fit but will not remove the data.

\end{fulllineitems}

\index{deactivate\_molecule() (VoigtFit.DataSet method)}

\begin{fulllineitems}
\phantomsection\label{\detokenize{api:VoigtFit.DataSet.deactivate_molecule}}\pysiglinewithargsret{\sphinxbfcode{\sphinxupquote{deactivate\_molecule}}}{\emph{molecule}, \emph{band}}{}
Deactivate all lines for the given band of the given molecule.
To see the available molecular bands defined, see the manual pdf
or \sphinxcode{\sphinxupquote{VoigtFit.molecules}}.

\end{fulllineitems}

\index{delete\_component() (VoigtFit.DataSet method)}

\begin{fulllineitems}
\phantomsection\label{\detokenize{api:VoigtFit.DataSet.delete_component}}\pysiglinewithargsret{\sphinxbfcode{\sphinxupquote{delete\_component}}}{\emph{ion}, \emph{index}}{}
Remove component of the given \sphinxtitleref{ion} with the given \sphinxtitleref{index}.

\end{fulllineitems}

\index{find\_ion() (VoigtFit.DataSet method)}

\begin{fulllineitems}
\phantomsection\label{\detokenize{api:VoigtFit.DataSet.find_ion}}\pysiglinewithargsret{\sphinxbfcode{\sphinxupquote{find\_ion}}}{\emph{ion}}{}
Return a list of all line tags for a given ion.

\end{fulllineitems}

\index{find\_line() (VoigtFit.DataSet method)}

\begin{fulllineitems}
\phantomsection\label{\detokenize{api:VoigtFit.DataSet.find_line}}\pysiglinewithargsret{\sphinxbfcode{\sphinxupquote{find\_line}}}{\emph{line\_tag}}{}
Look up the fitting {\hyperref[\detokenize{api:regions.Region}]{\sphinxcrossref{\sphinxcode{\sphinxupquote{Region}}}}} for a given \sphinxstyleemphasis{line tag}.
\begin{quote}\begin{description}
\item[{Parameters}] \leavevmode
\sphinxstylestrong{line\_tag} : str
\begin{quote}

The line tag of the line whose region will be returned.
\end{quote}

\item[{Returns}] \leavevmode
\sphinxstylestrong{regions\_of\_line} : list of {\hyperref[\detokenize{api:regions.Region}]{\sphinxcrossref{\sphinxcode{\sphinxupquote{Region}}}}}
\begin{quote}

A list of the fitting regions containing the given line.
This can be more than one region in case of overlapping or multiple spectra.
\end{quote}

\end{description}\end{quote}

\end{fulllineitems}

\index{fit() (VoigtFit.DataSet method)}

\begin{fulllineitems}
\phantomsection\label{\detokenize{api:VoigtFit.DataSet.fit}}\pysiglinewithargsret{\sphinxbfcode{\sphinxupquote{fit}}}{\emph{rebin=1}, \emph{verbose=True}, \emph{plot=False}, \emph{**kwargs}}{}
Fit the absorption lines using chi-square minimization.
\begin{quote}\begin{description}
\item[{Parameters}] \leavevmode
\sphinxstylestrong{rebin} : int   {[}default = 1{]}
\begin{quote}

Rebin data by a factor \sphinxstyleemphasis{rebin} before fitting.
\end{quote}

\sphinxstylestrong{verbose} : bool   {[}default = True{]}
\begin{quote}

This will print the fit results to terminal.
\end{quote}

\sphinxstylestrong{plot} : bool   {[}default = False{]}
\begin{quote}

This will make the best-fit solution show up in a new window.
\end{quote}

\sphinxstylestrong{**kwargs}
\begin{quote}

Keyword arguments are derived from the \sphinxhref{https://docs.scipy.org/doc/scipy/reference/tutorial/optimize.html}{scipy.optimize} minimization methods.
The default method is \sphinxhref{https://docs.scipy.org/doc/scipy/reference/generated/scipy.optimize.leastsq.html}{leastsq}, used in \sphinxhref{https://lmfit.github.io/lmfit-py/}{lmfit}.
This can be changed using the \sphinxtitleref{method} keyword.
See documentation in \sphinxhref{https://lmfit.github.io/lmfit-py/}{lmfit} and \sphinxhref{https://docs.scipy.org/doc/scipy/reference/tutorial/optimize.html}{scipy.optimize}.
\end{quote}

\item[{Returns}] \leavevmode
\sphinxstylestrong{popt} : \sphinxhref{https://lmfit.github.io/lmfit-py/fitting.html\#lmfit.minimizer.MinimizerResult}{lmfit.MinimizerResult}
\begin{quote}

The minimzer results from \sphinxhref{https://lmfit.github.io/lmfit-py/}{lmfit} containing best-fit parameters
and fit details, e.g., exit status and reduced chi squared.
See documentation for \sphinxhref{https://lmfit.github.io/lmfit-py/}{lmfit}.
\end{quote}

\sphinxstylestrong{chi2} : float
\begin{quote}

The chi squared value of the best-fit. Note that this value is \sphinxstylestrong{not}
the reduced chi squared. This value, and the number of degrees of freedom,
are available under the \sphinxtitleref{popt} object.
\end{quote}

\end{description}\end{quote}

\end{fulllineitems}

\index{fix\_structure() (VoigtFit.DataSet method)}

\begin{fulllineitems}
\phantomsection\label{\detokenize{api:VoigtFit.DataSet.fix_structure}}\pysiglinewithargsret{\sphinxbfcode{\sphinxupquote{fix\_structure}}}{\emph{ion=None}}{}
Fix the velocity structure, that is, the redshifts and the b-parameters.
\begin{quote}\begin{description}
\item[{Parameters}] \leavevmode
\sphinxstylestrong{ion} : str   {[}default = None{]}
\begin{quote}

The ion for which the structure should be fixed.
If \sphinxtitleref{None} is given, the structure is fixed for all ions.
\end{quote}

\end{description}\end{quote}

\end{fulllineitems}

\index{free\_structure() (VoigtFit.DataSet method)}

\begin{fulllineitems}
\phantomsection\label{\detokenize{api:VoigtFit.DataSet.free_structure}}\pysiglinewithargsret{\sphinxbfcode{\sphinxupquote{free\_structure}}}{\emph{ion=None}}{}
Free the velocity structure, that is, the redshifts and the b-parameters.
\begin{quote}\begin{description}
\item[{Parameters}] \leavevmode
\sphinxstylestrong{ion} : str   {[}default = None{]}
\begin{quote}

The ion for which the structure should be released.
If \sphinxtitleref{None} is given, the structure is released for all ions.
\end{quote}

\end{description}\end{quote}

\end{fulllineitems}

\index{get\_lines\_for\_ion() (VoigtFit.DataSet method)}

\begin{fulllineitems}
\phantomsection\label{\detokenize{api:VoigtFit.DataSet.get_lines_for_ion}}\pysiglinewithargsret{\sphinxbfcode{\sphinxupquote{get\_lines\_for\_ion}}}{\emph{ion}}{}
Return a list of {\hyperref[\detokenize{api:dataset.Line}]{\sphinxcrossref{\sphinxcode{\sphinxupquote{Line}}}}} objects
corresponding to the given \sphinxstyleemphasis{ion}.
\begin{quote}\begin{description}
\item[{Parameters}] \leavevmode
\sphinxstylestrong{ion} : str
\begin{quote}

The ion for the lines to get.
\end{quote}

\item[{Returns}] \leavevmode
\sphinxstylestrong{lines\_for\_ion} : list({\hyperref[\detokenize{api:dataset.Line}]{\sphinxcrossref{\sphinxcode{\sphinxupquote{Line}}}}})
\begin{quote}

List of Lines defined for the given \sphinxstyleemphasis{ion}.
\end{quote}

\end{description}\end{quote}

\end{fulllineitems}

\index{get\_name() (VoigtFit.DataSet method)}

\begin{fulllineitems}
\phantomsection\label{\detokenize{api:VoigtFit.DataSet.get_name}}\pysiglinewithargsret{\sphinxbfcode{\sphinxupquote{get\_name}}}{}{}
Returns the name of the DataSet.

\end{fulllineitems}

\index{get\_resolution() (VoigtFit.DataSet method)}

\begin{fulllineitems}
\phantomsection\label{\detokenize{api:VoigtFit.DataSet.get_resolution}}\pysiglinewithargsret{\sphinxbfcode{\sphinxupquote{get\_resolution}}}{\emph{line\_tag}, \emph{verbose=False}}{}
Return the spectral resolution for the fitting {\hyperref[\detokenize{api:regions.Region}]{\sphinxcrossref{\sphinxcode{\sphinxupquote{Region}}}}}
where the line with the given \sphinxtitleref{line\_tag} is defined.
\begin{quote}\begin{description}
\item[{Parameters}] \leavevmode
\sphinxstylestrong{line\_tag} : str
\begin{quote}

The line-tag for the line to look up: e.g., “FeII\_2374”
\end{quote}

\sphinxstylestrong{verbose} : bool   {[}default = False{]}
\begin{quote}

If \sphinxtitleref{True}, print the returned spectral resolution to std out.
\end{quote}

\item[{Returns}] \leavevmode
\sphinxstylestrong{resolutions} : list of float
\begin{quote}

A list of the spectral resolution of the fitting regions
where the given line is defined.
\end{quote}

\end{description}\end{quote}

\end{fulllineitems}

\index{has\_ion() (VoigtFit.DataSet method)}

\begin{fulllineitems}
\phantomsection\label{\detokenize{api:VoigtFit.DataSet.has_ion}}\pysiglinewithargsret{\sphinxbfcode{\sphinxupquote{has\_ion}}}{\emph{ion}, \emph{active\_only=False}}{}
Return True if the dataset has lines defined for the given ion.

\end{fulllineitems}

\index{has\_line() (VoigtFit.DataSet method)}

\begin{fulllineitems}
\phantomsection\label{\detokenize{api:VoigtFit.DataSet.has_line}}\pysiglinewithargsret{\sphinxbfcode{\sphinxupquote{has\_line}}}{\emph{line\_tag}, \emph{active\_only=False}}{}
Return True if the given line is defined.

\end{fulllineitems}

\index{interactive\_components() (VoigtFit.DataSet method)}

\begin{fulllineitems}
\phantomsection\label{\detokenize{api:VoigtFit.DataSet.interactive_components}}\pysiglinewithargsret{\sphinxbfcode{\sphinxupquote{interactive\_components}}}{\emph{line\_tag}}{}
Define components interactively for a given ion. The components will be defined on the
basis of the given line for that ion. If the line is defined in several spectra
then the interactive window will show up for each.
Running the interactive mode more times for different transitions of the same ion
will append the components to the structure.
If no components should be added, then simply click \sphinxtitleref{enter} to terminate the process
for the given transition.
\begin{quote}\begin{description}
\item[{Parameters}] \leavevmode
\sphinxstylestrong{line\_tag} : str
\begin{quote}

Line tag for the line belonging to the ion for which components should be defined.
\end{quote}

\end{description}\end{quote}
\paragraph{Notes}

This will launch an interactive plot showing the fitting region of the given line.
The user can then click on the positions of the components which. At the end, the
redshifts and estimated column densities are printed to terminal. The b-parameter
is assumed to be unresolved, i.e., taken from the resolution.

\end{fulllineitems}

\index{load\_components\_from\_file() (VoigtFit.DataSet method)}

\begin{fulllineitems}
\phantomsection\label{\detokenize{api:VoigtFit.DataSet.load_components_from_file}}\pysiglinewithargsret{\sphinxbfcode{\sphinxupquote{load\_components\_from\_file}}}{\emph{fname}}{}
Load best-fit parameters from an output file \sphinxtitleref{fname}.

\end{fulllineitems}

\index{mask\_line() (VoigtFit.DataSet method)}

\begin{fulllineitems}
\phantomsection\label{\detokenize{api:VoigtFit.DataSet.mask_line}}\pysiglinewithargsret{\sphinxbfcode{\sphinxupquote{mask\_line}}}{\emph{line\_tag}, \emph{reset=True}, \emph{mask=None}, \emph{telluric=True}}{}
Define exclusion masks for the fitting region of a given line.
Note that the masked regions are exclusion regions and will not be used for the fit.
If components have been defined, these will be shown as vertical lines.
\begin{quote}\begin{description}
\item[{Parameters}] \leavevmode
\sphinxstylestrong{line\_tag} : str
\begin{quote}

Line tag for the \sphinxcode{\sphinxupquote{Line}} whose
{\hyperref[\detokenize{api:regions.Region}]{\sphinxcrossref{\sphinxcode{\sphinxupquote{Region}}}}} should be masked.
\end{quote}

\sphinxstylestrong{reset} : bool   {[}default = True{]}
\begin{quote}

If \sphinxtitleref{True}, clear the mask before defining a new mask.
\end{quote}

\sphinxstylestrong{mask} : array\_like, shape (n)   {[}default = None{]}
\begin{quote}

If the mask is given, it must be a boolean array of the same length
as the region flux, err, and wl arrays.
Passing a mask this was supresses the interactive masking process.
\end{quote}

\sphinxstylestrong{telluric} : bool   {[}default = True{]}
\begin{quote}

If \sphinxtitleref{True}, a telluric absorption template and sky emission template
is shown for reference.
\end{quote}

\end{description}\end{quote}

\end{fulllineitems}

\index{normalize\_line() (VoigtFit.DataSet method)}

\begin{fulllineitems}
\phantomsection\label{\detokenize{api:VoigtFit.DataSet.normalize_line}}\pysiglinewithargsret{\sphinxbfcode{\sphinxupquote{normalize\_line}}}{\emph{line\_tag}, \emph{norm\_method='spline'}}{}
Normalize or re-normalize a given line
\begin{quote}\begin{description}
\item[{Parameters}] \leavevmode
\sphinxstylestrong{line\_tag} : str
\begin{quote}

Line tag of the line whose fitting region should be normalized.
\end{quote}

\sphinxstylestrong{norm\_method} : str   {[}default = ‘spline’{]}
\begin{quote}

Normalization method used for the interactive continuum fit.
Should be on of: {[}“spline”, “linear”{]}
\end{quote}

\end{description}\end{quote}

\end{fulllineitems}

\index{plot\_fit() (VoigtFit.DataSet method)}

\begin{fulllineitems}
\phantomsection\label{\detokenize{api:VoigtFit.DataSet.plot_fit}}\pysiglinewithargsret{\sphinxbfcode{\sphinxupquote{plot\_fit}}}{\emph{rebin=1}, \emph{fontsize=12}, \emph{xmin=None}, \emph{xmax=None}, \emph{max\_rows=4}, \emph{ymin=None}, \emph{ymax=None}, \emph{filename=None}, \emph{subsample\_profile=1}, \emph{npad=50}, \emph{loc='left'}, \emph{highlight\_props=None}, \emph{residuals=True}, \emph{norm\_resid=False}, \emph{default\_props=\{\}}, \emph{element\_props=\{\}}, \emph{legend=True}, \emph{label\_all\_ions=False}, \emph{xunit='vel'}}{}
Plot \sphinxstyleemphasis{all} the absorption lines and the best-fit profiles.
For details, see \sphinxcode{\sphinxupquote{VoigtFit.output.plot\_all\_lines()}}.

\end{fulllineitems}

\index{plot\_line() (VoigtFit.DataSet method)}

\begin{fulllineitems}
\phantomsection\label{\detokenize{api:VoigtFit.DataSet.plot_line}}\pysiglinewithargsret{\sphinxbfcode{\sphinxupquote{plot\_line}}}{\emph{line\_tag}, \emph{index=0}, \emph{plot\_fit=False}, \emph{loc='left'}, \emph{rebin=1}, \emph{nolabels=False}, \emph{axis=None}, \emph{fontsize=12}, \emph{xmin=None}, \emph{xmax=None}, \emph{ymin=None}, \emph{ymax=None}, \emph{show=True}, \emph{subsample\_profile=1}, \emph{npad=50}, \emph{residuals=True}, \emph{norm\_resid=False}, \emph{legend=True}, \emph{default\_props=\{\}}, \emph{element\_props=\{\}}, \emph{highlight\_props=None}, \emph{label\_all\_ions=False}, \emph{xunit='velocity'}}{}
Plot a single fitting {\hyperref[\detokenize{api:regions.Region}]{\sphinxcrossref{\sphinxcode{\sphinxupquote{Region}}}}}
containing the line corresponding to the given \sphinxtitleref{line\_tag}.
For details, see {\hyperref[\detokenize{api:output.plot_single_line}]{\sphinxcrossref{\sphinxcode{\sphinxupquote{output.plot\_single\_line()}}}}}.

\end{fulllineitems}

\index{prepare\_dataset() (VoigtFit.DataSet method)}

\begin{fulllineitems}
\phantomsection\label{\detokenize{api:VoigtFit.DataSet.prepare_dataset}}\pysiglinewithargsret{\sphinxbfcode{\sphinxupquote{prepare\_dataset}}}{\emph{norm=True}, \emph{mask=True}, \emph{verbose=True}, \emph{active\_only=False}, \emph{force\_clean=False}}{}
Prepare the data for fitting. This function sets up the parameter structure,
and handles the normalization and masking of fitting regions.
\begin{quote}\begin{description}
\item[{Parameters}] \leavevmode
\sphinxstylestrong{norm} : bool   {[}default = True{]}
\begin{quote}

Opens an interactive window to let the user normalize each region
using the defined \sphinxcode{\sphinxupquote{norm\_method}}.
\end{quote}

\sphinxstylestrong{mask} : bool   {[}default = True{]}
\begin{quote}

Opens an interactive window to let the user mask each fitting region.
\end{quote}

\sphinxstylestrong{verbose} : bool   {[}default = True{]}
\begin{quote}

If this is set, the code will print small info statements during the run.
\end{quote}

\sphinxstylestrong{force\_clean} : bool   {[}default = False{]}
\begin{quote}

If this is True, components for inactive elements will be removed.
\end{quote}

\item[{Returns}] \leavevmode
bool
\begin{quote}

The function returns \sphinxtitleref{True} when the dataset has passed all the steps.
If one step fails, the function returns \sphinxtitleref{False}.
The \sphinxcode{\sphinxupquote{ready2fit}} attribute of the dataset is also
updated accordingly.
\end{quote}

\end{description}\end{quote}

\end{fulllineitems}

\index{print\_cont\_parameters() (VoigtFit.DataSet method)}

\begin{fulllineitems}
\phantomsection\label{\detokenize{api:VoigtFit.DataSet.print_cont_parameters}}\pysiglinewithargsret{\sphinxbfcode{\sphinxupquote{print\_cont\_parameters}}}{}{}
Print Chebyshev coefficients for the continuum fit.

\end{fulllineitems}

\index{print\_metallicity() (VoigtFit.DataSet method)}

\begin{fulllineitems}
\phantomsection\label{\detokenize{api:VoigtFit.DataSet.print_metallicity}}\pysiglinewithargsret{\sphinxbfcode{\sphinxupquote{print\_metallicity}}}{\emph{logNHI}, \emph{err=0.1}}{}
Print the total column densities for each element relative to
HI in Solar units.

\end{fulllineitems}

\index{print\_results() (VoigtFit.DataSet method)}

\begin{fulllineitems}
\phantomsection\label{\detokenize{api:VoigtFit.DataSet.print_results}}\pysiglinewithargsret{\sphinxbfcode{\sphinxupquote{print\_results}}}{\emph{velocity=True}, \emph{elements='all'}, \emph{systemic=None}}{}
Print the best fit parameters.
\begin{quote}\begin{description}
\item[{Parameters}] \leavevmode
\sphinxstylestrong{velocity} : bool   {[}default = True{]}
\begin{quote}

If \sphinxtitleref{True}, show the relative velocities of each component instead of redshifts.
\end{quote}

\sphinxstylestrong{elements} : list(str)   {[}default = ‘all’{]}
\begin{quote}

A list of elements for which to show parameters.
\end{quote}

\sphinxstylestrong{systemic} : float   {[}default = None{]}
\begin{quote}

The systemic redshift used as reference for the relative velocities.
\end{quote}

\end{description}\end{quote}

\end{fulllineitems}

\index{print\_total() (VoigtFit.DataSet method)}

\begin{fulllineitems}
\phantomsection\label{\detokenize{api:VoigtFit.DataSet.print_total}}\pysiglinewithargsret{\sphinxbfcode{\sphinxupquote{print\_total}}}{}{}
Print the total column densities of all components.

\end{fulllineitems}

\index{remove\_fine\_lines() (VoigtFit.DataSet method)}

\begin{fulllineitems}
\phantomsection\label{\detokenize{api:VoigtFit.DataSet.remove_fine_lines}}\pysiglinewithargsret{\sphinxbfcode{\sphinxupquote{remove\_fine\_lines}}}{\emph{line\_tag}, \emph{levels=None}}{}
Remove lines associated to a given fine-structure complex.
\begin{quote}\begin{description}
\item[{Parameters}] \leavevmode
\sphinxstylestrong{line\_tag} : str
\begin{quote}

The line tag of the ground state transition to remove.
\end{quote}

\sphinxstylestrong{levels} : str, list(str)   {[}default = None{]}
\begin{quote}

The levels of the fine-structure complexes to remove, with “a” referring
to the first excited level, “b” is the second, etc..
Several levels can be given at once as a list: {[}‘a’, ‘b’{]}
or as a concatenated string: ‘abc’.
By default, all levels are included.
\end{quote}

\end{description}\end{quote}

\end{fulllineitems}

\index{remove\_line() (VoigtFit.DataSet method)}

\begin{fulllineitems}
\phantomsection\label{\detokenize{api:VoigtFit.DataSet.remove_line}}\pysiglinewithargsret{\sphinxbfcode{\sphinxupquote{remove\_line}}}{\emph{line\_tag}}{}
Remove an absorption line from the DataSet. If this is the last line in a fitting region
the given region will be eliminated, and if this is the last line of a given ion,
then the components will be eliminated for that ion.
\begin{quote}\begin{description}
\item[{Parameters}] \leavevmode
\sphinxstylestrong{line\_tag} : str
\begin{quote}

Line tag of the transition that should be removed.
\end{quote}

\end{description}\end{quote}

\end{fulllineitems}

\index{remove\_molecule() (VoigtFit.DataSet method)}

\begin{fulllineitems}
\phantomsection\label{\detokenize{api:VoigtFit.DataSet.remove_molecule}}\pysiglinewithargsret{\sphinxbfcode{\sphinxupquote{remove\_molecule}}}{\emph{molecule}, \emph{band}}{}
Remove all lines for the given band of the given molecule.

\end{fulllineitems}

\index{reset\_all\_regions() (VoigtFit.DataSet method)}

\begin{fulllineitems}
\phantomsection\label{\detokenize{api:VoigtFit.DataSet.reset_all_regions}}\pysiglinewithargsret{\sphinxbfcode{\sphinxupquote{reset\_all\_regions}}}{}{}
Reset the data in all {\hyperref[\detokenize{api:regions.Region}]{\sphinxcrossref{\sphinxcode{\sphinxupquote{Regions}}}}}
defined in the DataSet to use the raw input data.

\end{fulllineitems}

\index{reset\_components() (VoigtFit.DataSet method)}

\begin{fulllineitems}
\phantomsection\label{\detokenize{api:VoigtFit.DataSet.reset_components}}\pysiglinewithargsret{\sphinxbfcode{\sphinxupquote{reset\_components}}}{\emph{ion=None}}{}
Reset component structure for a given ion.
\begin{quote}\begin{description}
\item[{Parameters}] \leavevmode
\sphinxstylestrong{ion} : str   {[}default = None{]}
\begin{quote}

The ion for which to reset the components: e.g., ‘FeII’, ‘HI’, ‘CIa’, etc.
If {\color{red}\bfseries{}{}`}None{}`is given, \sphinxstyleemphasis{all} components for \sphinxstyleemphasis{all} ions will be reset.
\end{quote}

\end{description}\end{quote}

\end{fulllineitems}

\index{reset\_region() (VoigtFit.DataSet method)}

\begin{fulllineitems}
\phantomsection\label{\detokenize{api:VoigtFit.DataSet.reset_region}}\pysiglinewithargsret{\sphinxbfcode{\sphinxupquote{reset\_region}}}{\emph{reg}}{}
Reset the data in a given {\hyperref[\detokenize{api:regions.Region}]{\sphinxcrossref{\sphinxcode{\sphinxupquote{regions.Region}}}}} to use the raw input data.

\end{fulllineitems}

\index{save() (VoigtFit.DataSet method)}

\begin{fulllineitems}
\phantomsection\label{\detokenize{api:VoigtFit.DataSet.save}}\pysiglinewithargsret{\sphinxbfcode{\sphinxupquote{save}}}{\emph{filename=None}, \emph{verbose=False}}{}
Save the DataSet to file using the HDF5 format.

\end{fulllineitems}

\index{save\_cont\_parameters\_to\_file() (VoigtFit.DataSet method)}

\begin{fulllineitems}
\phantomsection\label{\detokenize{api:VoigtFit.DataSet.save_cont_parameters_to_file}}\pysiglinewithargsret{\sphinxbfcode{\sphinxupquote{save\_cont\_parameters\_to\_file}}}{\emph{filename}}{}
Save the best-fit continuum parameters to ASCII table output.
\begin{quote}\begin{description}
\item[{Parameters}] \leavevmode
\sphinxstylestrong{filename} : str   {[}default = None{]}
\begin{quote}

If \sphinxtitleref{None}, the \sphinxcode{\sphinxupquote{name}} attribute will be used.
\end{quote}

\end{description}\end{quote}

\end{fulllineitems}

\index{save\_fit\_regions() (VoigtFit.DataSet method)}

\begin{fulllineitems}
\phantomsection\label{\detokenize{api:VoigtFit.DataSet.save_fit_regions}}\pysiglinewithargsret{\sphinxbfcode{\sphinxupquote{save\_fit\_regions}}}{\emph{filename=None}, \emph{individual=False}, \emph{path=''}}{}
Save the fitting regions to ASCII table output.
The format is as follows:
(wavelength , normalized flux , normalized error , best-fit profile , mask)
\begin{quote}\begin{description}
\item[{Parameters}] \leavevmode
\sphinxstylestrong{filename} : str   {[}default = None{]}
\begin{quote}

Filename for the fitting regions.
If \sphinxtitleref{None}, the \sphinxcode{\sphinxupquote{name}} attribute will be used.
\end{quote}

\sphinxstylestrong{individual} : bool   {[}default = False{]}
\begin{quote}

Save the fitting regions to individual files.
By default all regions are concatenated into one file.
\end{quote}

\sphinxstylestrong{path} : str   {[}default = ‘’{]}
\begin{quote}

Specify a path to prepend to the filename in order to save output to a given
directory or path. Can be given both as relative or absolute path.
If the path doesn’t end in \sphinxtitleref{/} it will be appended automatically.
The final filename will be:
\begin{quote}

\sphinxtitleref{path/} + \sphinxtitleref{filename} {[}+ \sphinxtitleref{\_regN}{]} + \sphinxtitleref{.reg}
\end{quote}
\end{quote}

\end{description}\end{quote}

\end{fulllineitems}

\index{save\_parameters() (VoigtFit.DataSet method)}

\begin{fulllineitems}
\phantomsection\label{\detokenize{api:VoigtFit.DataSet.save_parameters}}\pysiglinewithargsret{\sphinxbfcode{\sphinxupquote{save\_parameters}}}{\emph{filename}}{}
Save the best-fit parameters to ASCII table output.
\begin{quote}\begin{description}
\item[{Parameters}] \leavevmode
\sphinxstylestrong{filename} : str
\begin{quote}

Filename for the fit parameter file.
\end{quote}

\end{description}\end{quote}

\end{fulllineitems}

\index{set\_name() (VoigtFit.DataSet method)}

\begin{fulllineitems}
\phantomsection\label{\detokenize{api:VoigtFit.DataSet.set_name}}\pysiglinewithargsret{\sphinxbfcode{\sphinxupquote{set\_name}}}{\emph{name}}{}
Set the name of the DataSet. This parameter is used when saving the dataset.

\end{fulllineitems}

\index{set\_resolution() (VoigtFit.DataSet method)}

\begin{fulllineitems}
\phantomsection\label{\detokenize{api:VoigtFit.DataSet.set_resolution}}\pysiglinewithargsret{\sphinxbfcode{\sphinxupquote{set\_resolution}}}{\emph{res}, \emph{line\_tag=None}, \emph{verbose=True}}{}
Set the spectral resolution in km/s for the {\hyperref[\detokenize{api:regions.Region}]{\sphinxcrossref{\sphinxcode{\sphinxupquote{Region}}}}}
containing the line with the given \sphinxtitleref{line\_tag}. If multiple spectra are fitted
simultaneously, this method will set the same resolution for \sphinxstyleemphasis{all} spectra.
If \sphinxtitleref{line\_tag} is not given, the resolution will be set for \sphinxstyleemphasis{all} regions,
including the raw data chunks defined in \sphinxcode{\sphinxupquote{VoigtFit.DataSet.data}}!

Note \textendash{} If not all data chunks have the same resolution, this method
should be used with caution. It is advised to check the spectral resolution beforehand
and only update the appropriate regions using a for-loop.

\end{fulllineitems}

\index{set\_systemic\_redshift() (VoigtFit.DataSet method)}

\begin{fulllineitems}
\phantomsection\label{\detokenize{api:VoigtFit.DataSet.set_systemic_redshift}}\pysiglinewithargsret{\sphinxbfcode{\sphinxupquote{set\_systemic\_redshift}}}{\emph{z\_sys}}{}
Update the systemic redshift of the dataset

\end{fulllineitems}

\index{sum\_components() (VoigtFit.DataSet method)}

\begin{fulllineitems}
\phantomsection\label{\detokenize{api:VoigtFit.DataSet.sum_components}}\pysiglinewithargsret{\sphinxbfcode{\sphinxupquote{sum\_components}}}{\emph{ions}, \emph{components}}{}
Calculate the total column density for the given \sphinxtitleref{components}
of the given \sphinxtitleref{ion}.
\begin{quote}\begin{description}
\item[{Parameters}] \leavevmode
\sphinxstylestrong{ions} : str or list(str)
\begin{quote}

List of ions or a single ion for which to calculate the total
column density.
\end{quote}

\sphinxstylestrong{components} : list(int)
\begin{quote}

List of integers corresponding to the indeces of the components
to sum over.
\end{quote}

\item[{Returns}] \leavevmode
\sphinxstylestrong{total\_logN} : dict()
\begin{quote}

Dictionary containing the log of total column density for each ion.
\end{quote}

\sphinxstylestrong{total\_logN\_err} : dict()
\begin{quote}

Dictionary containing the error on the log of total column density
for each ion.
\end{quote}

\end{description}\end{quote}

\end{fulllineitems}

\index{velocity\_plot() (VoigtFit.DataSet method)}

\begin{fulllineitems}
\phantomsection\label{\detokenize{api:VoigtFit.DataSet.velocity_plot}}\pysiglinewithargsret{\sphinxbfcode{\sphinxupquote{velocity\_plot}}}{\emph{**kwargs}}{}
Create a velocity plot, showing all the fitting regions defined, in order to compare
different lines and to identify blends and contamination.

\end{fulllineitems}


\end{fulllineitems}



\subsection{class \sphinxstylestrong{Line}}
\label{\detokenize{api:class-line}}\index{Line (class in dataset)}

\begin{fulllineitems}
\phantomsection\label{\detokenize{api:dataset.Line}}\pysiglinewithargsret{\sphinxbfcode{\sphinxupquote{class }}\sphinxcode{\sphinxupquote{dataset.}}\sphinxbfcode{\sphinxupquote{Line}}}{\emph{tag}, \emph{active=True}}{}
Line object containing atomic data for the given transition.
Only the line\_tag is passed, the rest of the information is
looked up in the atomic database.
\paragraph{Attributes}
\begin{description}
\item[{tag}] \leavevmode{[}str{]}
The line tag for the line, e.g., “FeII\_2374”

\item[{ion}] \leavevmode{[}str{]}
The ion for the line; The ion for “FeII\_2374” is “FeII”.

\item[{element}] \leavevmode{[}str{]}
Equal to \sphinxcode{\sphinxupquote{ion}} for backwards compatibility.

\item[{l0}] \leavevmode{[}float{]}
Rest-frame resonant wavelength of the transition.
Unit: Angstrom.

\item[{f}] \leavevmode{[}float{]}
The oscillator strength for the transition.

\item[{gam}] \leavevmode{[}float{]}
The radiation damping constant or Einstein coefficient.

\item[{mass}] \leavevmode{[}float{]}
The atomic mass in atomic mass units.

\item[{active}] \leavevmode{[}bool   {[}default = True{]}{]}
The state of the line in the dataset. Only active lines will
be included in the fit.

\end{description}
\paragraph{Methods}


\begin{savenotes}\sphinxatlongtablestart\begin{longtable}{p{0.5\linewidth}p{0.5\linewidth}}
\hline

\endfirsthead

\multicolumn{2}{c}%
{\makebox[0pt]{\sphinxtablecontinued{\tablename\ \thetable{} -- continued from previous page}}}\\
\hline

\endhead

\hline
\multicolumn{2}{r}{\makebox[0pt][r]{\sphinxtablecontinued{Continued on next page}}}\\
\endfoot

\endlastfoot

{\hyperref[\detokenize{api:dataset.Line.get_properties}]{\sphinxcrossref{\sphinxcode{\sphinxupquote{get\_properties}}}}}()
&
Return the principal atomic constants for the transition: \sphinxstyleemphasis{l0}, \sphinxstyleemphasis{f}, and \sphinxstyleemphasis{gam}.
\\
\hline
{\hyperref[\detokenize{api:dataset.Line.set_active}]{\sphinxcrossref{\sphinxcode{\sphinxupquote{set\_active}}}}}()
&
Set the line active; include the line in the fit.
\\
\hline
{\hyperref[\detokenize{api:dataset.Line.set_inactive}]{\sphinxcrossref{\sphinxcode{\sphinxupquote{set\_inactive}}}}}()
&
Set the line inactive; exclude the line in the fit.
\\
\hline
\end{longtable}\sphinxatlongtableend\end{savenotes}
\index{get\_properties() (dataset.Line method)}

\begin{fulllineitems}
\phantomsection\label{\detokenize{api:dataset.Line.get_properties}}\pysiglinewithargsret{\sphinxbfcode{\sphinxupquote{get\_properties}}}{}{}
Return the principal atomic constants for the transition: \sphinxstyleemphasis{l0}, \sphinxstyleemphasis{f}, and \sphinxstyleemphasis{gam}.

\end{fulllineitems}

\index{set\_active() (dataset.Line method)}

\begin{fulllineitems}
\phantomsection\label{\detokenize{api:dataset.Line.set_active}}\pysiglinewithargsret{\sphinxbfcode{\sphinxupquote{set\_active}}}{}{}
Set the line active; include the line in the fit.

\end{fulllineitems}

\index{set\_inactive() (dataset.Line method)}

\begin{fulllineitems}
\phantomsection\label{\detokenize{api:dataset.Line.set_inactive}}\pysiglinewithargsret{\sphinxbfcode{\sphinxupquote{set\_inactive}}}{}{}
Set the line inactive; exclude the line in the fit.

\end{fulllineitems}


\end{fulllineitems}



\subsection{class \sphinxstylestrong{Region}}
\label{\detokenize{api:class-region}}\index{Region (class in regions)}

\begin{fulllineitems}
\phantomsection\label{\detokenize{api:regions.Region}}\pysiglinewithargsret{\sphinxbfcode{\sphinxupquote{class }}\sphinxcode{\sphinxupquote{regions.}}\sphinxbfcode{\sphinxupquote{Region}}}{\emph{velspan}, \emph{specID}, \emph{line=None}}{}
A Region contains the fitting data, exclusion mask and line information.
The class is instantiated with the velocity span, \sphinxtitleref{velspan}, and a spectral ID
pointing to the raw data chunk from \sphinxtitleref{DataSet.data},
and can include a {\hyperref[\detokenize{api:dataset.Line}]{\sphinxcrossref{\sphinxcode{\sphinxupquote{dataset.Line}}}}} instance for the first line
belonging to the region.
\paragraph{Attributes}
\begin{description}
\item[{velspan}] \leavevmode{[}float{]}
The velocity range to used for the fitting region.

\item[{lines}] \leavevmode{[}list({\hyperref[\detokenize{api:dataset.Line}]{\sphinxcrossref{\sphinxcode{\sphinxupquote{dataset.Line}}}}}){]}
A list of Lines defined in the region.

\item[{label}] \leavevmode{[}str{]}
A LaTeX label describing the lines in the region for plotting purposes.

\item[{res}] \leavevmode{[}float{]}
Spectral resolution of the region in km/s.

\item[{wl}] \leavevmode{[}array\_like, shape (N){]}
Data array of wavelengths in Ångstrøm.

\item[{flux}] \leavevmode{[}array\_like, shape (N){]}
Data array of fluxes (normalized if \sphinxcode{\sphinxupquote{normalized}} is \sphinxtitleref{True}).

\item[{err}] \leavevmode{[}array\_like, shape (N){]}
Array of uncertainties for each flux element.

\item[{normalized}] \leavevmode{[}bool{]}
\sphinxtitleref{True} if the data in the region are normlized.

\item[{mask}] \leavevmode{[}array\_like, shape (N){]}
Exclusion mask for the region:
0/\sphinxtitleref{False} = pixel is \sphinxstyleemphasis{not} included in the fit.
1/\sphinxtitleref{True} = pixel is included in the fit.

\item[{new\_mask}] \leavevmode{[}bool{]}
Internal parameter for {\hyperref[\detokenize{api:VoigtFit.DataSet.prepare_dataset}]{\sphinxcrossref{\sphinxcode{\sphinxupquote{VoigtFit.DataSet.prepare\_dataset()}}}}}.
If \sphinxtitleref{True}, an interactive masking process will be initiated in the
preparation stage.

\item[{cont\_err}] \leavevmode{[}float{]}
An estimate of the uncertainty in the continuum fit.

\item[{specID}] \leavevmode{[}str{]}
A spectral identifier to point back to the raw data chunk.

\end{description}
\paragraph{Methods}


\begin{savenotes}\sphinxatlongtablestart\begin{longtable}{p{0.5\linewidth}p{0.5\linewidth}}
\hline

\endfirsthead

\multicolumn{2}{c}%
{\makebox[0pt]{\sphinxtablecontinued{\tablename\ \thetable{} -- continued from previous page}}}\\
\hline

\endhead

\hline
\multicolumn{2}{r}{\makebox[0pt][r]{\sphinxtablecontinued{Continued on next page}}}\\
\endfoot

\endlastfoot

{\hyperref[\detokenize{api:regions.Region.add_data_to_region}]{\sphinxcrossref{\sphinxcode{\sphinxupquote{add\_data\_to\_region}}}}}(data\_chunk, cutout)
&
Define the spectral data for the fitting region.
\\
\hline
{\hyperref[\detokenize{api:regions.Region.add_line}]{\sphinxcrossref{\sphinxcode{\sphinxupquote{add\_line}}}}}(line)
&
Add a new {\hyperref[\detokenize{api:dataset.Line}]{\sphinxcrossref{\sphinxcode{\sphinxupquote{dataset.Line}}}}} to the fitting region.
\\
\hline
{\hyperref[\detokenize{api:regions.Region.clear_mask}]{\sphinxcrossref{\sphinxcode{\sphinxupquote{clear\_mask}}}}}()
&
Clear the already defined mask in the region.
\\
\hline
{\hyperref[\detokenize{api:regions.Region.define_mask}]{\sphinxcrossref{\sphinxcode{\sphinxupquote{define\_mask}}}}}({[}z, dataset, telluric{]})
&
Use an interactive window to define the mask for the region.
\\
\hline
{\hyperref[\detokenize{api:regions.Region.generate_label}]{\sphinxcrossref{\sphinxcode{\sphinxupquote{generate\_label}}}}}({[}active\_only{]})
&
Automatically generate a descriptive label for the region.
\\
\hline
{\hyperref[\detokenize{api:regions.Region.has_active_lines}]{\sphinxcrossref{\sphinxcode{\sphinxupquote{has\_active\_lines}}}}}()
&
Return \sphinxtitleref{True} is at least one line in the region is active.
\\
\hline
{\hyperref[\detokenize{api:regions.Region.has_line}]{\sphinxcrossref{\sphinxcode{\sphinxupquote{has\_line}}}}}(line\_tag)
&
Return \sphinxtitleref{True} if a line with the given \sphinxtitleref{line\_tag} is defined in the region.
\\
\hline
{\hyperref[\detokenize{api:regions.Region.is_normalized}]{\sphinxcrossref{\sphinxcode{\sphinxupquote{is\_normalized}}}}}()
&
Return \sphinxtitleref{True} if the region data is normalized.
\\
\hline
{\hyperref[\detokenize{api:regions.Region.normalize}]{\sphinxcrossref{\sphinxcode{\sphinxupquote{normalize}}}}}({[}plot, norm\_method{]})
&
Normalize the region if the data are not already normalized.
\\
\hline
{\hyperref[\detokenize{api:regions.Region.remove_line}]{\sphinxcrossref{\sphinxcode{\sphinxupquote{remove\_line}}}}}(line\_tag)
&
Remove absorption line with the given \sphinxtitleref{line\_tag} from the region.
\\
\hline
{\hyperref[\detokenize{api:regions.Region.set_label}]{\sphinxcrossref{\sphinxcode{\sphinxupquote{set\_label}}}}}(text)
&
Set descriptive text label for the given region.
\\
\hline
\sphinxcode{\sphinxupquote{set\_mask}}(mask)
&

\\
\hline
{\hyperref[\detokenize{api:regions.Region.unpack}]{\sphinxcrossref{\sphinxcode{\sphinxupquote{unpack}}}}}()
&
Return the data of the region (wl, flux, error, mask)
\\
\hline
\end{longtable}\sphinxatlongtableend\end{savenotes}
\index{add\_data\_to\_region() (regions.Region method)}

\begin{fulllineitems}
\phantomsection\label{\detokenize{api:regions.Region.add_data_to_region}}\pysiglinewithargsret{\sphinxbfcode{\sphinxupquote{add\_data\_to\_region}}}{\emph{data\_chunk}, \emph{cutout}}{}
Define the spectral data for the fitting region.
\begin{quote}\begin{description}
\item[{Parameters}] \leavevmode
\sphinxstylestrong{data\_chunk} : dict()
\begin{quote}

A \sphinxtitleref{data\_chunk} as defined in the data structure of {\hyperref[\detokenize{api:VoigtFit.DataSet.add_data}]{\sphinxcrossref{\sphinxcode{\sphinxupquote{DataSet.data}}}}}.
\end{quote}

\sphinxstylestrong{cutout} : bool array
\begin{quote}

A boolean array defining the subset of the \sphinxtitleref{data\_chunk} which makes up the fitting region.
\end{quote}

\end{description}\end{quote}

\end{fulllineitems}

\index{add\_line() (regions.Region method)}

\begin{fulllineitems}
\phantomsection\label{\detokenize{api:regions.Region.add_line}}\pysiglinewithargsret{\sphinxbfcode{\sphinxupquote{add\_line}}}{\emph{line}}{}
Add a new {\hyperref[\detokenize{api:dataset.Line}]{\sphinxcrossref{\sphinxcode{\sphinxupquote{dataset.Line}}}}} to the fitting region.

\end{fulllineitems}

\index{clear\_mask() (regions.Region method)}

\begin{fulllineitems}
\phantomsection\label{\detokenize{api:regions.Region.clear_mask}}\pysiglinewithargsret{\sphinxbfcode{\sphinxupquote{clear\_mask}}}{}{}
Clear the already defined mask in the region.

\end{fulllineitems}

\index{define\_mask() (regions.Region method)}

\begin{fulllineitems}
\phantomsection\label{\detokenize{api:regions.Region.define_mask}}\pysiglinewithargsret{\sphinxbfcode{\sphinxupquote{define\_mask}}}{\emph{z=None}, \emph{dataset=None}, \emph{telluric=True}}{}
Use an interactive window to define the mask for the region.
\begin{quote}\begin{description}
\item[{Parameters}] \leavevmode
\sphinxstylestrong{z} : float   {[}default = None{]}
\begin{quote}

If a redshift is given, the lines in the region are shown as vertical lines
at the given redshift.
\end{quote}

\sphinxstylestrong{dataset} : {\hyperref[\detokenize{api:VoigtFit.DataSet}]{\sphinxcrossref{\sphinxcode{\sphinxupquote{VoigtFit.DataSet}}}}}   {[}default = None{]}
\begin{quote}

A dataset with components defined for the lines in the region.
If a dataset is passed, the components of the lines in the region are shown.
\end{quote}

\sphinxstylestrong{telluric} : bool   {[}default = True{]}
\begin{quote}

Show telluric absorption and sky emission line templates during the masking.
\end{quote}

\end{description}\end{quote}

\end{fulllineitems}

\index{generate\_label() (regions.Region method)}

\begin{fulllineitems}
\phantomsection\label{\detokenize{api:regions.Region.generate_label}}\pysiglinewithargsret{\sphinxbfcode{\sphinxupquote{generate\_label}}}{\emph{active\_only=True}}{}
Automatically generate a descriptive label for the region.

\end{fulllineitems}

\index{has\_active\_lines() (regions.Region method)}

\begin{fulllineitems}
\phantomsection\label{\detokenize{api:regions.Region.has_active_lines}}\pysiglinewithargsret{\sphinxbfcode{\sphinxupquote{has\_active\_lines}}}{}{}
Return \sphinxtitleref{True} is at least one line in the region is active.

\end{fulllineitems}

\index{has\_line() (regions.Region method)}

\begin{fulllineitems}
\phantomsection\label{\detokenize{api:regions.Region.has_line}}\pysiglinewithargsret{\sphinxbfcode{\sphinxupquote{has\_line}}}{\emph{line\_tag}}{}
Return \sphinxtitleref{True} if a line with the given \sphinxtitleref{line\_tag} is defined in the region.

\end{fulllineitems}

\index{is\_normalized() (regions.Region method)}

\begin{fulllineitems}
\phantomsection\label{\detokenize{api:regions.Region.is_normalized}}\pysiglinewithargsret{\sphinxbfcode{\sphinxupquote{is\_normalized}}}{}{}
Return \sphinxtitleref{True} if the region data is normalized.

\end{fulllineitems}

\index{normalize() (regions.Region method)}

\begin{fulllineitems}
\phantomsection\label{\detokenize{api:regions.Region.normalize}}\pysiglinewithargsret{\sphinxbfcode{\sphinxupquote{normalize}}}{\emph{plot=True}, \emph{norm\_method='linear'}}{}
Normalize the region if the data are not already normalized.
Choose from two methods:
\begin{quote}
\begin{description}
\item[{1:  define left and right continuum regions}] \leavevmode
and fit a linear continuum.

\item[{2:  define the continuum as a range of points}] \leavevmode
and use spline interpolation to infer the
continuum.

\end{description}
\end{quote}

\end{fulllineitems}

\index{remove\_line() (regions.Region method)}

\begin{fulllineitems}
\phantomsection\label{\detokenize{api:regions.Region.remove_line}}\pysiglinewithargsret{\sphinxbfcode{\sphinxupquote{remove\_line}}}{\emph{line\_tag}}{}
Remove absorption line with the given \sphinxtitleref{line\_tag} from the region.

\end{fulllineitems}

\index{set\_label() (regions.Region method)}

\begin{fulllineitems}
\phantomsection\label{\detokenize{api:regions.Region.set_label}}\pysiglinewithargsret{\sphinxbfcode{\sphinxupquote{set\_label}}}{\emph{text}}{}
Set descriptive text label for the given region.

\end{fulllineitems}

\index{unpack() (regions.Region method)}

\begin{fulllineitems}
\phantomsection\label{\detokenize{api:regions.Region.unpack}}\pysiglinewithargsret{\sphinxbfcode{\sphinxupquote{unpack}}}{}{}
Return the data of the region (wl, flux, error, mask)

\end{fulllineitems}


\end{fulllineitems}



\subsection{module \sphinxstylestrong{voigt}}
\label{\detokenize{api:module-voigt}}\index{voigt (module)}
The module contains functions to evaluate the optical depth,
to convert this to observed transmission and to convolve the
observed spectrum with the instrumental profile.
\index{H() (in module voigt)}

\begin{fulllineitems}
\phantomsection\label{\detokenize{api:voigt.H}}\pysiglinewithargsret{\sphinxcode{\sphinxupquote{voigt.}}\sphinxbfcode{\sphinxupquote{H}}}{\emph{a}, \emph{x}}{}
Voigt Profile Approximation from T. Tepper-Garcia 2006, 2007.

\end{fulllineitems}

\index{Voigt() (in module voigt)}

\begin{fulllineitems}
\phantomsection\label{\detokenize{api:voigt.Voigt}}\pysiglinewithargsret{\sphinxcode{\sphinxupquote{voigt.}}\sphinxbfcode{\sphinxupquote{Voigt}}}{\emph{l}, \emph{l0}, \emph{f}, \emph{N}, \emph{b}, \emph{gam}, \emph{z=0}}{}
Calculate the optical depth Voigt profile.
\begin{quote}\begin{description}
\item[{Parameters}] \leavevmode
\sphinxstylestrong{l} : array\_like, shape (N)
\begin{quote}

Wavelength grid in Angstroms at which to evaluate the optical depth.
\end{quote}

\sphinxstylestrong{l0} : float
\begin{quote}

Rest frame transition wavelength in Angstroms.
\end{quote}

\sphinxstylestrong{f} : float
\begin{quote}

Oscillator strength.
\end{quote}

\sphinxstylestrong{N} : float
\begin{quote}

Column density in units of cm\textasciicircum{}-2.
\end{quote}

\sphinxstylestrong{b} : float
\begin{quote}

Velocity width of the Voigt profile in cm/s.
\end{quote}

\sphinxstylestrong{gam} : float
\begin{quote}

Radiation damping constant, or Einstein constant (A\_ul)
\end{quote}

\sphinxstylestrong{z} : float
\begin{quote}

The redshift of the observed wavelength grid \sphinxtitleref{l}.
\end{quote}

\item[{Returns}] \leavevmode
\sphinxstylestrong{tau} : array\_like, shape (N)
\begin{quote}

Optical depth array evaluated at the input grid wavelengths \sphinxtitleref{l}.
\end{quote}

\end{description}\end{quote}

\end{fulllineitems}

\index{evaluate\_continuum() (in module voigt)}

\begin{fulllineitems}
\phantomsection\label{\detokenize{api:voigt.evaluate_continuum}}\pysiglinewithargsret{\sphinxcode{\sphinxupquote{voigt.}}\sphinxbfcode{\sphinxupquote{evaluate\_continuum}}}{\emph{x}, \emph{pars}, \emph{reg\_num}}{}
Evaluate the continuum model using Chebyshev polynomials.
All regions are fitted with the same order of polynomials.
\begin{quote}\begin{description}
\item[{Parameters}] \leavevmode
\sphinxstylestrong{x} : array\_like, shape (N)
\begin{quote}

Input wavelength grid in Ångstrøm.
\end{quote}

\sphinxstylestrong{pars} : dict(\sphinxhref{https://lmfit.github.io/lmfit-py/parameters.html}{lmfit.Parameters})
\begin{quote}

An instance of \sphinxhref{https://lmfit.github.io/lmfit-py/parameters.html}{lmfit.Parameters} containing the Chebyshev
coefficients for each region.
\end{quote}

\sphinxstylestrong{reg\_num} : int
\begin{quote}

The region number, i.e., the index of the region in the list
\sphinxcode{\sphinxupquote{VoigtFit.DataSet.regions}}.
\end{quote}

\item[{Returns}] \leavevmode
\sphinxstylestrong{cont\_model} : array\_like, shape (N)
\begin{quote}

The continuum Chebyshev polynomial evaluated at the input wavelengths \sphinxtitleref{x}.
\end{quote}

\end{description}\end{quote}

\end{fulllineitems}

\index{evaluate\_profile() (in module voigt)}

\begin{fulllineitems}
\phantomsection\label{\detokenize{api:voigt.evaluate_profile}}\pysiglinewithargsret{\sphinxcode{\sphinxupquote{voigt.}}\sphinxbfcode{\sphinxupquote{evaluate\_profile}}}{\emph{x}, \emph{pars}, \emph{z\_sys}, \emph{lines}, \emph{components}, \emph{res}, \emph{dv=0.1}}{}
Evaluate the observed Voigt profile. The calculated optical depth, \sphinxtitleref{tau},
is converted to observed transmission, \sphinxtitleref{f}:
\begin{equation*}
\begin{split}f = e^{-\tau}\end{split}
\end{equation*}
The observed transmission is subsequently convolved with the instrumental
broadening profile assumed to be Gaussian with a full-width at half maximum
of res. The resolving power is assumed to be constant in velocity space.
\begin{quote}\begin{description}
\item[{Parameters}] \leavevmode
\sphinxstylestrong{x} : array\_like, shape (N)
\begin{quote}

Wavelength array in Ångstrøm on which to evaluate the profile.
\end{quote}

\sphinxstylestrong{pars} : dict(\sphinxhref{https://lmfit.github.io/lmfit-py/parameters.html}{lmfit.Parameters})
\begin{quote}

An instance of \sphinxhref{https://lmfit.github.io/lmfit-py/parameters.html}{lmfit.Parameters} containing the line parameters.
\end{quote}

\sphinxstylestrong{lines} : list({\hyperref[\detokenize{api:dataset.Line}]{\sphinxcrossref{\sphinxcode{\sphinxupquote{Line}}}}})
\begin{quote}

List of lines to evaluate. Should be a list of
{\hyperref[\detokenize{api:dataset.Line}]{\sphinxcrossref{\sphinxcode{\sphinxupquote{Line}}}}} objects.
\end{quote}

\sphinxstylestrong{components} : dict
\begin{quote}

Dictionary containing component data for the defined ions.
See \sphinxcode{\sphinxupquote{VoigtFit.DataSet.components}}.
\end{quote}

\sphinxstylestrong{res} : float
\begin{quote}

Spectral resolving power of the data in km/s  {[}= \sphinxstyleemphasis{c/R}{]}.
\end{quote}

\sphinxstylestrong{dv} : float  {[}default = 0.1{]}
\begin{quote}

Desired pixel size of subsampled profile grid in km/s.
\end{quote}

\item[{Returns}] \leavevmode
\sphinxstylestrong{profile\_obs} : array\_like, shape (N)
\begin{quote}

Observed line profile convolved with the instrument profile.
\end{quote}

\end{description}\end{quote}

\end{fulllineitems}



\subsection{module \sphinxstylestrong{output}}
\label{\detokenize{api:module-output}}\index{output (module)}\index{chunks() (in module output)}

\begin{fulllineitems}
\phantomsection\label{\detokenize{api:output.chunks}}\pysiglinewithargsret{\sphinxcode{\sphinxupquote{output.}}\sphinxbfcode{\sphinxupquote{chunks}}}{\emph{l}, \emph{n}}{}
Yield successive \sphinxtitleref{n}-sized chunks from \sphinxtitleref{l}.

\end{fulllineitems}

\index{create\_blank\_input() (in module output)}

\begin{fulllineitems}
\phantomsection\label{\detokenize{api:output.create_blank_input}}\pysiglinewithargsret{\sphinxcode{\sphinxupquote{output.}}\sphinxbfcode{\sphinxupquote{create\_blank\_input}}}{}{}
Create a blank template input parameter file.

\end{fulllineitems}

\index{mad() (in module output)}

\begin{fulllineitems}
\phantomsection\label{\detokenize{api:output.mad}}\pysiglinewithargsret{\sphinxcode{\sphinxupquote{output.}}\sphinxbfcode{\sphinxupquote{mad}}}{\emph{x}}{}
Calculate Median Absolute Deviation

\end{fulllineitems}

\index{plot\_H2() (in module output)}

\begin{fulllineitems}
\phantomsection\label{\detokenize{api:output.plot_H2}}\pysiglinewithargsret{\sphinxcode{\sphinxupquote{output.}}\sphinxbfcode{\sphinxupquote{plot\_H2}}}{\emph{dataset}, \emph{n\_rows=None}, \emph{xmin=None}, \emph{xmax=None}, \emph{ymin=-0.1}, \emph{ymax=2.5}, \emph{short\_labels=False}, \emph{rebin=1}, \emph{smooth=0}}{}
Generate plot for H2 absorption lines.
\begin{quote}\begin{description}
\item[{Parameters}] \leavevmode
\sphinxstylestrong{dataset} : {\hyperref[\detokenize{api:VoigtFit.DataSet}]{\sphinxcrossref{\sphinxcode{\sphinxupquote{VoigtFit.DataSet}}}}}
\begin{quote}

An instance of the class {\hyperref[\detokenize{api:VoigtFit.DataSet}]{\sphinxcrossref{\sphinxcode{\sphinxupquote{VoigtFit.DataSet}}}}} containing
the H2 lines to plot.
\end{quote}

\sphinxstylestrong{n\_rows} : int   {[}default = None{]}
\begin{quote}

Number of rows to show in figure.
If None, the number will be determined automatically.
\end{quote}

\sphinxstylestrong{xmin} : float
\begin{quote}

The lower x-limit in Å.
If nothing is given, the extent of the fit region is used.
\end{quote}

\sphinxstylestrong{xmax} : float
\begin{quote}

The upper x-limit in Å.
If nothing is given, the extent of the fit region is used.
\end{quote}

\sphinxstylestrong{ymin} : float   {[}default = -0.1{]}
\begin{quote}

The lower y-limit in normalized flux units.
\end{quote}

\sphinxstylestrong{ymax} : float   {[}default = 2.5{]}
\begin{quote}

The upper y-limit in normalized flux units.
\end{quote}

\sphinxstylestrong{rebin} : int   {[}defualt = 1{]}
\begin{quote}

Rebinning factor for the spectrum, default is no binning.
\end{quote}

\sphinxstylestrong{smooth} : float   {[}default = 0{]}
\begin{quote}

Width of Gaussian kernel for smoothing.
\end{quote}

\end{description}\end{quote}

\end{fulllineitems}

\index{plot\_all\_lines() (in module output)}

\begin{fulllineitems}
\phantomsection\label{\detokenize{api:output.plot_all_lines}}\pysiglinewithargsret{\sphinxcode{\sphinxupquote{output.}}\sphinxbfcode{\sphinxupquote{plot\_all\_lines}}}{\emph{dataset}, \emph{plot\_fit=True}, \emph{rebin=1}, \emph{fontsize=12}, \emph{xmin=None}, \emph{xmax=None}, \emph{ymin=None}, \emph{ymax=None}, \emph{max\_rows=4}, \emph{filename=None}, \emph{subsample\_profile=1}, \emph{npad=50}, \emph{residuals=True}, \emph{norm\_resid=False}, \emph{legend=True}, \emph{loc='left'}, \emph{show=True}, \emph{default\_props=\{\}}, \emph{element\_props=\{\}}, \emph{highlight\_props=None}, \emph{label\_all\_ions=False}, \emph{xunit='vel'}}{}
Plot all active absorption lines. This function is a wrapper of
{\hyperref[\detokenize{api:output.plot_single_line}]{\sphinxcrossref{\sphinxcode{\sphinxupquote{plot\_single\_line()}}}}}. For a complete description of input parameters,
see the documentation for {\hyperref[\detokenize{api:output.plot_single_line}]{\sphinxcrossref{\sphinxcode{\sphinxupquote{plot\_single\_line()}}}}}.
\begin{quote}\begin{description}
\item[{Parameters}] \leavevmode
\sphinxstylestrong{dataset} : {\hyperref[\detokenize{api:VoigtFit.DataSet}]{\sphinxcrossref{\sphinxcode{\sphinxupquote{VoigtFit.DataSet}}}}}
\begin{quote}

Instance of the {\hyperref[\detokenize{api:VoigtFit.DataSet}]{\sphinxcrossref{\sphinxcode{\sphinxupquote{VoigtFit.DataSet}}}}} class containing the line
regions to plot.
\end{quote}

\sphinxstylestrong{max\_rows} : int   {[}default = 4{]}
\begin{quote}

The maximum number of rows of figures.
Each row consists of two figure panels.
\end{quote}

\sphinxstylestrong{filename} : str
\begin{quote}

If a filename is given, the figures are saved to a pdf file.
\end{quote}

\end{description}\end{quote}

\end{fulllineitems}

\index{plot\_excitation() (in module output)}

\begin{fulllineitems}
\phantomsection\label{\detokenize{api:output.plot_excitation}}\pysiglinewithargsret{\sphinxcode{\sphinxupquote{output.}}\sphinxbfcode{\sphinxupquote{plot\_excitation}}}{\emph{dataset}, \emph{molecule}}{}
Plot the excitation diagram for a given \sphinxtitleref{molecule}

\end{fulllineitems}

\index{plot\_residual() (in module output)}

\begin{fulllineitems}
\phantomsection\label{\detokenize{api:output.plot_residual}}\pysiglinewithargsret{\sphinxcode{\sphinxupquote{output.}}\sphinxbfcode{\sphinxupquote{plot\_residual}}}{\emph{dataset}, \emph{line\_tag}, \emph{index=0}, \emph{rebin=1}, \emph{xmin=None}, \emph{xmax=None}, \emph{axis=None}}{}
Plot residuals for the best-fit to a given absorption line.
\begin{quote}\begin{description}
\item[{Parameters}] \leavevmode
\sphinxstylestrong{dataset} : class DataSet
\begin{quote}

An instance of DataSet class containing the line region to plot.
\end{quote}

\sphinxstylestrong{line\_tag} : str
\begin{quote}

The line tag of the line to show, e.g., ‘FeII\_2374’
\end{quote}

\sphinxstylestrong{index} : int   {[}default = 0{]}
\begin{quote}

The line index. When fitting the same line in multiple
spectra this indexed points to the index of the given region
to be plotted.
\end{quote}

\sphinxstylestrong{rebin: int}
\begin{quote}

Integer factor for rebinning the spectral data.
\end{quote}

\sphinxstylestrong{xmin} : float
\begin{quote}

The lower x-limit in relative velocity (km/s).
If nothing is given, the extent of the region is used.
\end{quote}

\sphinxstylestrong{xmax} : float
\begin{quote}

The upper x-limit in relative velocity (km/s).
If nothing is given, the extent of the region is used.
\end{quote}

\sphinxstylestrong{axis} : \sphinxhref{https://matplotlib.org/api/axes\_api.html}{matplotlib.axes.Axes}
\begin{quote}

The plotting axis of matplotlib.
If \sphinxtitleref{None} is given, a new figure and axis will be created.
\end{quote}

\end{description}\end{quote}

\end{fulllineitems}

\index{plot\_single\_line() (in module output)}

\begin{fulllineitems}
\phantomsection\label{\detokenize{api:output.plot_single_line}}\pysiglinewithargsret{\sphinxcode{\sphinxupquote{output.}}\sphinxbfcode{\sphinxupquote{plot\_single\_line}}}{\emph{dataset}, \emph{line\_tag}, \emph{index=0}, \emph{plot\_fit=False}, \emph{loc='left'}, \emph{rebin=1}, \emph{nolabels=False}, \emph{axis=None}, \emph{fontsize=12}, \emph{subsample\_profile=1}, \emph{xmin=None}, \emph{xmax=None}, \emph{ymin=None}, \emph{ymax=None}, \emph{show=True}, \emph{npad=50}, \emph{legend=True}, \emph{residuals=False}, \emph{norm\_resid=False}, \emph{default\_props=\{\}}, \emph{element\_props=\{\}}, \emph{highlight\_props=None}, \emph{label\_all\_ions=False}, \emph{xunit='velocity'}}{}
Plot a single absorption line.
\begin{quote}\begin{description}
\item[{Parameters}] \leavevmode
\sphinxstylestrong{dataset} : {\hyperref[\detokenize{api:VoigtFit.DataSet}]{\sphinxcrossref{\sphinxcode{\sphinxupquote{VoigtFit.DataSet}}}}}
\begin{quote}

Instance of the {\hyperref[\detokenize{api:VoigtFit.DataSet}]{\sphinxcrossref{\sphinxcode{\sphinxupquote{VoigtFit.DataSet}}}}} class containing the lines
\end{quote}

\sphinxstylestrong{line\_tag} : str
\begin{quote}

The line tag of the line to show, e.g., ‘FeII\_2374’
\end{quote}

\sphinxstylestrong{index} : int   {[}default = 0{]}
\begin{quote}

The line index. When fitting the same line in multiple
spectra this indexed points to the index of the given
region to be plotted.
\end{quote}

\sphinxstylestrong{plot\_fit} : bool   {[}default = False{]}
\begin{quote}

If \sphinxtitleref{True}, the best-fit profile will be shown
\end{quote}

\sphinxstylestrong{loc} : str   {[}default = ‘left’{]}
\begin{quote}

Places the line tag (right or left).
\end{quote}

\sphinxstylestrong{rebin} : int   {[}default = 1{]}
\begin{quote}

Rebinning factor for the spectrum
\end{quote}

\sphinxstylestrong{nolabels} : bool   {[}default = False{]}
\begin{quote}

If \sphinxtitleref{True}, show the axis x- and y-labels.
\end{quote}

\sphinxstylestrong{axis} : \sphinxhref{https://matplotlib.org/api/axes\_api.html}{matplotlib.axes.Axes}
\begin{quote}

The plotting axis of matplotlib.
If \sphinxtitleref{None} is given, a new figure and axis will be created.
\end{quote}

\sphinxstylestrong{fontsize} : int   {[}default = 12{]}
\begin{quote}

The fontsize of text labels.
\end{quote}

\sphinxstylestrong{xmin} : float
\begin{quote}

The lower x-limit in relative velocity (km/s).
If nothing is given, the extent of the region is used.
\end{quote}

\sphinxstylestrong{xmax} : float
\begin{quote}

The upper x-limit in relative velocity (km/s).
If nothing is given, the extent of the region is used.
\end{quote}

\sphinxstylestrong{ymin} : float   {[}default = None{]}
\begin{quote}

The lower y-limit in normalized flux units.
Default is determined from the data.
\end{quote}

\sphinxstylestrong{ymax} : float   {[}default = None{]}
\begin{quote}

The upper y-limit in normalized flux units.
Default is determined from the data.
\end{quote}

\sphinxstylestrong{show} : bool   {[}default = True{]}
\begin{quote}

Show the figure.
\end{quote}

\sphinxstylestrong{subsample\_profile} : int   {[}default = 1{]}
\begin{quote}

Subsampling factor to calculate the profile on a finer grid than
the data sampling.
By default the profile is evaluated on the same grid as the data.
\end{quote}

\sphinxstylestrong{npad} : int   {[}default = 50{]}
\begin{quote}

Padding added to the synthetic profile before convolution.
This removes end artefacts from the \sphinxtitleref{FFT} routine.
\end{quote}

\sphinxstylestrong{residuals} : bool   {[}default = False{]}
\begin{quote}

Add a panel above the absorption line to show the residuals of the fit.
\end{quote}

\sphinxstylestrong{norm\_resid} : bool   {[}default = False{]}
\begin{quote}

Show normalized residuals.
\end{quote}

\sphinxstylestrong{legend} : bool   {[}default = True{]}
\begin{quote}

Show line labels as axis legend.
\end{quote}

\sphinxstylestrong{default\_props} : dict
\begin{quote}

Dictionary of transition tick marker properties. The dictionary is
passed to matplotlib.axes.Axes.axvline to control color, linewidth,
linestyle, etc.. Two additional keywords can be defined:
The keyword \sphinxtitleref{text} is a string that will be printed above or below
each tick mark for each element.
The keyword \sphinxtitleref{loc} controls the placement of the tick mark text
for the transistions, and must be one either ‘above’ or ‘below’.
\end{quote}

\sphinxstylestrong{element\_props} : dict
\begin{quote}

Dictionary of properties for individual elements.
Each element defines a dictionary with individual properties following
the format for \sphinxtitleref{default\_props}.
\begin{quote}
\begin{description}
\item[{Ex: \sphinxcode{\sphinxupquote{element\_props=\{'SiII': \{'color': 'red', 'lw': 1.5\},}}}] \leavevmode
\sphinxcode{\sphinxupquote{'FeII': \{'ls': '-{-}', 'alpha': 0.2\}\}}}

\end{description}
\end{quote}

This will set the color and linewidth of the tick marks
of SiII transitions and the linestyle and alpha-parameter
of FeII transitions.
\end{quote}

\sphinxstylestrong{highlight\_props} : dict/list   {[}default = None{]}
\begin{quote}

A dictionary of \sphinxtitleref{ions} (e.g., “FeII”, “CIa”, etc.) used to calculate
a separate profile for this subset of ions. Each \sphinxtitleref{ion} as a keyword
must specify a dictionary which can change individual properties for
the given \sphinxtitleref{ion}. Similar to \sphinxtitleref{element\_props}.
If an empty dictionary is given, the default parameters will be used.
Alternatively, a list of \sphinxtitleref{ions} can be given to use default properties
for all \sphinxtitleref{ions}.
\begin{quote}

Ex: \sphinxcode{\sphinxupquote{highlight\_props=\{'SiII':\{\}, 'FeII':\{'color': 'blue'\}\}}}
\end{quote}

This will highlight SiII transitions with default highlight
properties, and FeII transistions with a user specified color.
\begin{quote}

Ex: \sphinxcode{\sphinxupquote{highlight\_props={[}'SiII', 'FeII'{]}}}
\end{quote}

This will highlight SiII and FeII transitions using default
highlight properties.
\end{quote}

\sphinxstylestrong{label\_all\_ions} : bool   {[}default = False{]}
\begin{quote}

Show labels for all \sphinxtitleref{ions} defined.
The labels will appear above the component tick marks.
\end{quote}

\sphinxstylestrong{xunit} : string   {[}default = {]}’velocity’{]}
\begin{quote}

The unit of the x-axis, must be either ‘velocity’ or ‘wavelength’.
Shortenings are acceptable too, e.g., ‘vel’/’v’ or ‘wave’/’wl’.
\end{quote}

\end{description}\end{quote}

\end{fulllineitems}

\index{print\_T\_model\_pars() (in module output)}

\begin{fulllineitems}
\phantomsection\label{\detokenize{api:output.print_T_model_pars}}\pysiglinewithargsret{\sphinxcode{\sphinxupquote{output.}}\sphinxbfcode{\sphinxupquote{print\_T\_model\_pars}}}{\emph{dataset}, \emph{thermal\_model}, \emph{filename=None}}{}
Print the turbulence and temperature parameters for physical model.

\end{fulllineitems}

\index{print\_cont\_parameters() (in module output)}

\begin{fulllineitems}
\phantomsection\label{\detokenize{api:output.print_cont_parameters}}\pysiglinewithargsret{\sphinxcode{\sphinxupquote{output.}}\sphinxbfcode{\sphinxupquote{print\_cont\_parameters}}}{\emph{dataset}}{}
Print the Chebyshev coefficients of the continuum normalization.

\end{fulllineitems}

\index{print\_metallicity() (in module output)}

\begin{fulllineitems}
\phantomsection\label{\detokenize{api:output.print_metallicity}}\pysiglinewithargsret{\sphinxcode{\sphinxupquote{output.}}\sphinxbfcode{\sphinxupquote{print\_metallicity}}}{\emph{dataset}, \emph{params}, \emph{logNHI}, \emph{err=0.1}}{}
Print the metallicity derived from different species.
This will add the column densities for all components of a given ion.
\begin{quote}\begin{description}
\item[{Parameters}] \leavevmode
\sphinxstylestrong{dataset} : {\hyperref[\detokenize{api:VoigtFit.DataSet}]{\sphinxcrossref{\sphinxcode{\sphinxupquote{VoigtFit.DataSet}}}}}
\begin{quote}

An instance of the {\hyperref[\detokenize{api:VoigtFit.DataSet}]{\sphinxcrossref{\sphinxcode{\sphinxupquote{VoigtFit.DataSet}}}}} class containing
the definition of data and absorption lines.
\end{quote}

\sphinxstylestrong{params} : \sphinxhref{https://lmfit.github.io/lmfit-py/parameters.html}{lmfit.Parameters}
\begin{quote}

Output parameter dictionary, e.g., \sphinxcode{\sphinxupquote{VoigtFit.DataSet.best\_fit}}.
See \sphinxhref{https://lmfit.github.io/lmfit-py/}{lmfit} for details.
\end{quote}

\sphinxstylestrong{logNHI} : float
\begin{quote}

Column density of neutral hydrogen.
\end{quote}

\sphinxstylestrong{err} : float   {[}default = 0.1{]}
\begin{quote}

Uncertainty (1-sigma) on \sphinxtitleref{logNHI}.
\end{quote}

\sphinxstylestrong{.. \_lmfit: https://lmfit.github.io/lmfit-py/index.html}

\end{description}\end{quote}

\end{fulllineitems}

\index{print\_results() (in module output)}

\begin{fulllineitems}
\phantomsection\label{\detokenize{api:output.print_results}}\pysiglinewithargsret{\sphinxcode{\sphinxupquote{output.}}\sphinxbfcode{\sphinxupquote{print\_results}}}{\emph{dataset}, \emph{params}, \emph{elements='all'}, \emph{velocity=True}, \emph{systemic=0}}{}
Print the parameters of the best-fit.
\begin{quote}\begin{description}
\item[{Parameters}] \leavevmode
\sphinxstylestrong{dataset} : {\hyperref[\detokenize{api:VoigtFit.DataSet}]{\sphinxcrossref{\sphinxcode{\sphinxupquote{VoigtFit.DataSet}}}}}
\begin{quote}

An instance of the {\hyperref[\detokenize{api:VoigtFit.DataSet}]{\sphinxcrossref{\sphinxcode{\sphinxupquote{VoigtFit.DataSet}}}}} class containing
the line region to plot.
\end{quote}

\sphinxstylestrong{params} : \sphinxhref{https://lmfit.github.io/lmfit-py/parameters.html}{lmfit.Parameters}
\begin{quote}

Output parameter dictionary, e.g., \sphinxcode{\sphinxupquote{dataset.Dataset.best\_fit}}.
See \sphinxhref{https://lmfit.github.io/lmfit-py/}{lmfit} for details.
\end{quote}

\sphinxstylestrong{elements} : list(str)   {[}default = ‘all’{]}
\begin{quote}

A list of ions for which to print the best-fit parameters.
By default all ions are shown.
\end{quote}

\sphinxstylestrong{velocity} : bool   {[}default = True{]}
\begin{quote}

Show the components in relative velocity or redshift.
\end{quote}

\sphinxstylestrong{systemic} : float   {[}default = 0{]}
\begin{quote}

Systemic redshift used to calculate the relative velocities.
By default the systemic redshift defined in the dataset is used.
\end{quote}

\sphinxstylestrong{.. \_lmfit: https://lmfit.github.io/lmfit-py/index.html}

\end{description}\end{quote}

\end{fulllineitems}

\index{print\_total() (in module output)}

\begin{fulllineitems}
\phantomsection\label{\detokenize{api:output.print_total}}\pysiglinewithargsret{\sphinxcode{\sphinxupquote{output.}}\sphinxbfcode{\sphinxupquote{print\_total}}}{\emph{dataset}}{}
Print the total column densities of all species. This will sum \sphinxstyleemphasis{all}
the components of each ion. The uncertainty on the total column density
is calculated using random resampling within the errors of each component.

\end{fulllineitems}

\index{rebin\_bool\_array() (in module output)}

\begin{fulllineitems}
\phantomsection\label{\detokenize{api:output.rebin_bool_array}}\pysiglinewithargsret{\sphinxcode{\sphinxupquote{output.}}\sphinxbfcode{\sphinxupquote{rebin\_bool\_array}}}{\emph{x}, \emph{n}}{}
Rebin input boolean array \sphinxtitleref{x} by an integer factor of \sphinxtitleref{n}.

\end{fulllineitems}

\index{rebin\_spectrum() (in module output)}

\begin{fulllineitems}
\phantomsection\label{\detokenize{api:output.rebin_spectrum}}\pysiglinewithargsret{\sphinxcode{\sphinxupquote{output.}}\sphinxbfcode{\sphinxupquote{rebin\_spectrum}}}{\emph{wl}, \emph{spec}, \emph{err}, \emph{n}, \emph{method='mean'}}{}
Rebin input spectrum by a factor of \sphinxtitleref{n}.
Method is either \sphinxstyleemphasis{mean} or \sphinxstyleemphasis{median}, default is mean.
\begin{quote}\begin{description}
\item[{Parameters}] \leavevmode
\sphinxstylestrong{wl} : array\_like, shape (N)
\begin{quote}

Input wavelength array.
\end{quote}

\sphinxstylestrong{spec} : array\_like, shape (N)
\begin{quote}

Input flux array.
\end{quote}

\sphinxstylestrong{err} : array\_like, shape (N)
\begin{quote}

Input error array.
\end{quote}

\sphinxstylestrong{n} : int
\begin{quote}

Integer rebinning factor.
\end{quote}

\sphinxstylestrong{method} : str   {[}default = ‘mean’{]}
\begin{quote}

Rebin method, either ‘mean’ or ‘median’.
\end{quote}

\item[{Returns}] \leavevmode
\sphinxstylestrong{wl\_r} : array\_like, shape (M)
\begin{quote}

Rebinned wavelength array, the new shape will be N/n.
\end{quote}

\sphinxstylestrong{spec\_r} : array\_like, shape (M)
\begin{quote}

Rebinned flux array, the new shape will be N/n.
\end{quote}

\sphinxstylestrong{err\_r} : array\_like, shape (M)
\begin{quote}

Rebinned error array, the new shape will be N/n.
\end{quote}

\end{description}\end{quote}

\end{fulllineitems}

\index{save\_cont\_parameters\_to\_file() (in module output)}

\begin{fulllineitems}
\phantomsection\label{\detokenize{api:output.save_cont_parameters_to_file}}\pysiglinewithargsret{\sphinxcode{\sphinxupquote{output.}}\sphinxbfcode{\sphinxupquote{save\_cont\_parameters\_to\_file}}}{\emph{dataset}, \emph{filename}}{}
Save Chebyshev coefficients to file.

\end{fulllineitems}

\index{save\_fit\_regions() (in module output)}

\begin{fulllineitems}
\phantomsection\label{\detokenize{api:output.save_fit_regions}}\pysiglinewithargsret{\sphinxcode{\sphinxupquote{output.}}\sphinxbfcode{\sphinxupquote{save\_fit\_regions}}}{\emph{dataset}, \emph{filename}, \emph{individual=False}, \emph{path=''}}{}
Save fit regions to ASCII file.
\begin{quote}\begin{description}
\item[{Parameters}] \leavevmode
\sphinxstylestrong{filename} : str
\begin{quote}

Filename to be used. If the filename already exists, it will be overwritten.
A \sphinxtitleref{.reg} file extension will automatically be append if not present already.
\end{quote}

\sphinxstylestrong{individual} : bool   {[}default = False{]}
\begin{quote}

If \sphinxtitleref{True}, save each fitting region to a separate file.
The individual filenames will be the basename given as \sphinxtitleref{filename}
with \sphinxtitleref{\_regN} appended, where \sphinxtitleref{N} is an integer referring to the region number.
\end{quote}

\sphinxstylestrong{path} : str   {[}default = ‘’{]}
\begin{quote}

Specify a path to prepend to the filename in order to save output to a given
directory or path. Can be given both as relative or absolute path.
If the directory does not exist, it will be created.
The final filename will be:
\begin{quote}

\sphinxtitleref{path/} + \sphinxtitleref{filename} {[}+ \sphinxtitleref{\_regN}{]} + \sphinxtitleref{.reg}
\end{quote}
\end{quote}

\end{description}\end{quote}

\end{fulllineitems}

\index{save\_parameters\_to\_file() (in module output)}

\begin{fulllineitems}
\phantomsection\label{\detokenize{api:output.save_parameters_to_file}}\pysiglinewithargsret{\sphinxcode{\sphinxupquote{output.}}\sphinxbfcode{\sphinxupquote{save\_parameters\_to\_file}}}{\emph{dataset}, \emph{filename}}{}
Save best-fit parameters to file.

\end{fulllineitems}

\index{show\_H2\_bands() (in module output)}

\begin{fulllineitems}
\phantomsection\label{\detokenize{api:output.show_H2_bands}}\pysiglinewithargsret{\sphinxcode{\sphinxupquote{output.}}\sphinxbfcode{\sphinxupquote{show\_H2\_bands}}}{\emph{ax}, \emph{z}, \emph{bands}, \emph{Jmax}, \emph{color='blue'}, \emph{short\_labels=False}}{}
Add molecular H2 band identifications to a given matplotlib axis.
\begin{quote}\begin{description}
\item[{Parameters}] \leavevmode
\sphinxstylestrong{ax} : \sphinxhref{https://matplotlib.org/api/axes\_api.html}{matplotlib.axes.Axes}
\begin{quote}

The axis instance to decorate with H2 line identifications.
\end{quote}

\sphinxstylestrong{z} : float
\begin{quote}

The redshift of the H2 system.
\end{quote}

\sphinxstylestrong{bands} : list(str)
\begin{quote}

A list of molecular bands of H2.
Ex: {[}‘BX(0-0)’, ‘BX(1-0)’{]}
\end{quote}

\sphinxstylestrong{Jmax} : int or list(int)
\begin{quote}

The highest rotational J-level to include in the identification.
A list of Jmax for each band can be passed as well.
Lines are currently only defined up to J=7.
\end{quote}

\end{description}\end{quote}

\end{fulllineitems}

\index{sum\_components() (in module output)}

\begin{fulllineitems}
\phantomsection\label{\detokenize{api:output.sum_components}}\pysiglinewithargsret{\sphinxcode{\sphinxupquote{output.}}\sphinxbfcode{\sphinxupquote{sum\_components}}}{\emph{dataset}, \emph{ion}, \emph{components}}{}
Calculate the total abundance for given \sphinxtitleref{components} of the given \sphinxtitleref{ion}.
\begin{quote}\begin{description}
\item[{Parameters}] \leavevmode
\sphinxstylestrong{dataset} : {\hyperref[\detokenize{api:VoigtFit.DataSet}]{\sphinxcrossref{\sphinxcode{\sphinxupquote{VoigtFit.DataSet}}}}}
\begin{quote}

An instance of the {\hyperref[\detokenize{api:VoigtFit.DataSet}]{\sphinxcrossref{\sphinxcode{\sphinxupquote{VoigtFit.DataSet}}}}} class containing
the definition of data and absorption lines.
\end{quote}

\sphinxstylestrong{ion} : str
\begin{quote}

Ion for which to calculate the summed abundance.
\end{quote}

\sphinxstylestrong{components} : list(int)
\begin{quote}

List of indeces of the components to sum over.
\end{quote}

\item[{Returns}] \leavevmode
\sphinxstylestrong{total\_logN} : float
\begin{quote}

The 10-base log of total column density.
\end{quote}

\sphinxstylestrong{total\_logN\_err} : float
\begin{quote}

The error on the 10-base log of total column density.
\end{quote}

\end{description}\end{quote}

\end{fulllineitems}



\chapter{Examples}
\label{\detokenize{index:examples}}

\section{Results of Physical Model}
\label{\detokenize{physical_model_results:results-of-physical-model}}\label{\detokenize{physical_model_results::doc}}
The physical model assumes that the b-parameter of each component
can be described by a common turbulent term, \(b_{turb}\), and
a thermal term, \(\sqrt{2 k_B T / m}\):
\begin{quote}
\begin{equation*}
\begin{split}b_{eff}^2 = b_{turb}^2 + \frac{2 k_B T}{m} ,\end{split}
\end{equation*}\end{quote}

where \sphinxstyleemphasis{m} is the atomic mass of the given ion.
The model thus expands the regular set of parameters to include
a temperature and turbulent velocity for each component.
For the exact implementation of this model, see the script
\sphinxcode{\sphinxupquote{physical\_model.py}}.


\subsection{Best fit parameters}
\label{\detokenize{physical_model_results:best-fit-parameters}}
Fitting the simulated test data with two components we recover
the following parameters:

\fvset{hllines={, ,}}%
\begin{sphinxVerbatim}[commandchars=\\\{\}]
CrII : 2026, 2056, 2062, 2066
    z = 0.003299    b = 4.75 \(\pm\) 0.01  log(N) = 12.581 \(\pm\) 0.034
    z = 0.003599    b = 5.44 \(\pm\) 0.01  log(N) = 12.167 \(\pm\) 0.084

SiII : 1260, 1304, 1526, 1808
    z = 0.003299    b = 4.84 \(\pm\) 0.01  log(N) = 14.461 \(\pm\) 0.006
    z = 0.003599    b = 5.67 \(\pm\) 0.01  log(N) = 13.955 \(\pm\) 0.003

FeII : 1260, 1608, 1611, 2249, 2260, 2344, 2374, 2382
    z = 0.003299    b = 4.74 \(\pm\) 0.01  log(N) = 14.065 \(\pm\) 0.004
    z = 0.003599    b = 5.42 \(\pm\) 0.01  log(N) = 13.589 \(\pm\) 0.003

CII : 1036, 1334
    z = 0.003299    b = 5.11 \(\pm\) 0.03  log(N) = 15.345 \(\pm\) 0.016
    z = 0.003599    b = 6.28 \(\pm\) 0.02  log(N) = 14.763 \(\pm\) 0.006

OI : 1039, 1302, 1355
    z = 0.003299    b = 4.99 \(\pm\) 0.02  log(N) = 15.709 \(\pm\) 0.010
    z = 0.003599    b = 6.02 \(\pm\) 0.02  log(N) = 15.173 \(\pm\) 0.005

ZnII : 2026, 2062
    z = 0.003299    b = 4.72 \(\pm\) 0.01  log(N) = 12.007 \(\pm\) 0.030
    z = 0.003599    b = 5.38 \(\pm\) 0.00  log(N) = 11.577 \(\pm\) 0.082

SII : 1250, 1253, 1259
    z = 0.003299    b = 4.82 \(\pm\) 0.01  log(N) = 14.176 \(\pm\) 0.012
    z = 0.003599    b = 5.61 \(\pm\) 0.01  log(N) = 13.701 \(\pm\) 0.035
\end{sphinxVerbatim}

The best-fit is plotted and saved to PDF file:  \sphinxcode{\sphinxupquote{test\_2comp.pdf}}


\subsection{Total Abundances}
\label{\detokenize{physical_model_results:total-abundances}}
The total abundances from the fit are:

\fvset{hllines={, ,}}%
\begin{sphinxVerbatim}[commandchars=\\\{\}]
logN(CrII) = 12.72 \(\pm\) 0.03
logN(SiII) = 14.58 \(\pm\) 0.01
logN(FeII) = 14.19 \(\pm\) 0.01
logN(CII) = 15.45 \(\pm\) 0.01
logN(OI) = 15.82 \(\pm\) 0.01
logN(ZnII) = 12.15 \(\pm\) 0.03
logN(SII) = 14.30 \(\pm\) 0.01
\end{sphinxVerbatim}


\subsection{Physical Parameters}
\label{\detokenize{physical_model_results:physical-parameters}}

\begin{savenotes}\sphinxattablestart
\centering
\begin{tabulary}{\linewidth}[t]{|T|T|T|}
\hline
\sphinxstyletheadfamily 
Comp. No:
&\sphinxstyletheadfamily 
Temperature {[}K{]}
&\sphinxstyletheadfamily 
Turbulence {[}km/s{]}
\\
\hline
0
&
3380 \(\pm\) 268
&
4.63 \(\pm\) 0.02
\\
\hline
1
&
9280 \(\pm\) 206
&
5.16 \(\pm\) 0.0002
\\
\hline
\end{tabulary}
\par
\sphinxattableend\end{savenotes}


\subsection{The Input Parameters}
\label{\detokenize{physical_model_results:the-input-parameters}}
Element Column Densities:

\fvset{hllines={, ,}}%
\begin{sphinxVerbatim}[commandchars=\\\{\}]
 \PYG{n}{ion}   \PYG{n}{comp1}  \PYG{n}{comp2}  \PYG{n}{total}
 \PYG{n}{AlII}  \PYG{l+m+mf}{12.95}  \PYG{l+m+mf}{12.47}  \PYG{l+m+mf}{13.07}
\PYG{n}{AlIII}  \PYG{l+m+mf}{12.38}  \PYG{l+m+mf}{11.90}  \PYG{l+m+mf}{12.50}
 \PYG{n}{ZnII}  \PYG{l+m+mf}{12.02}  \PYG{l+m+mf}{11.54}  \PYG{l+m+mf}{12.14}
 \PYG{n}{SiII}  \PYG{l+m+mf}{14.43}  \PYG{l+m+mf}{13.95}  \PYG{l+m+mf}{14.55}
   \PYG{n}{OI}  \PYG{l+m+mf}{15.64}  \PYG{l+m+mf}{15.16}  \PYG{l+m+mf}{15.76}
 \PYG{n}{FeII}  \PYG{l+m+mf}{14.07}  \PYG{l+m+mf}{13.59}  \PYG{l+m+mf}{14.19}
  \PYG{n}{CII}  \PYG{l+m+mf}{15.20}  \PYG{l+m+mf}{14.72}  \PYG{l+m+mf}{15.32}
  \PYG{n}{SII}  \PYG{l+m+mf}{14.17}  \PYG{l+m+mf}{13.69}  \PYG{l+m+mf}{14.29}
 \PYG{n}{CrII}  \PYG{l+m+mf}{12.58}  \PYG{l+m+mf}{12.10}  \PYG{l+m+mf}{12.70}
\end{sphinxVerbatim}


\subsection{Velocity Information:}
\label{\detokenize{physical_model_results:velocity-information}}

\begin{savenotes}\sphinxattablestart
\centering
\begin{tabulary}{\linewidth}[t]{|T|T|T|T|}
\hline
\sphinxstyletheadfamily 
Comp. No:
&\sphinxstyletheadfamily 
z
&\sphinxstyletheadfamily 
b\_turb
&\sphinxstyletheadfamily 
T
\\
\hline\sphinxstyletheadfamily &\sphinxstyletheadfamily &\sphinxstyletheadfamily 
{[}km/s{]}
&\sphinxstyletheadfamily 
{[}K{]}
\\
\hline
1
&
0.003300
&
4.60
&
4370
\\
\hline
2
&
0.003600
&
5.20
&
8600
\\
\hline
\end{tabulary}
\par
\sphinxattableend\end{savenotes}


\renewcommand{\indexname}{Python Module Index}
\begin{sphinxtheindex}
\def\bigletter#1{{\Large\sffamily#1}\nopagebreak\vspace{1mm}}
\bigletter{o}
\item {\sphinxstyleindexentry{output}}\sphinxstyleindexpageref{api:\detokenize{module-output}}
\indexspace
\bigletter{v}
\item {\sphinxstyleindexentry{voigt}}\sphinxstyleindexpageref{api:\detokenize{module-voigt}}
\end{sphinxtheindex}

\renewcommand{\indexname}{Index}
\printindex
\end{document}